\documentclass[12pt]{report}
\usepackage{natbib}
\usepackage{setspace}
    \doublespacing
\usepackage{graphicx}
\usepackage[top=1.5in, bottom=1.5in, left=1.5in, right=1.5in]{geometry}
\usepackage{booktabs}
\usepackage{longtable}
\usepackage{dcolumn}


\begin{document}

\title{Dissertation: Mass Media and the Domestic Politics of Economic Globalization}
\author{Justin Murphy, PhD Candidate, Temple University}



\maketitle


\textbf{Email:} \\
j.murphy@soton.ac.uk \\
\textbf{Address:} \\
Justin Murphy \\
Politics \& International Relations \\
Social Sciences \\
University of Southampton \\
Southampton SO17 1BJ, \\
United Kingdom


\chapter{Introduction}

The central argument of this dissertation is that the mass media have
played a critical but misunderstood role in the variety of national
political responses to economic globalization around the world since
the 1960s. Specifically, the studies collected here suggest that the
mass media have played a variety of \emph{anti-democratic} roles in
national liberalization processes since the 1960s, in ways which have
gone largely unnoticed by political scientists. It is the view of
this dissertation that variation across domestic media environments
can help us to explain broad empirical patterns in global and comparative
political economy, such as why some states in some periods have liberalized
in broadly inclusive ways, (for instance, the ``embedded
liberalism" of post-war Europe (\citealt{Ruggie:1982wx}),
while others have liberalized far less inclusively (for instance,
the many neoliberal economic openings of the late 1980s). Although
this is the general pattern to which the following studies testify,
each study disaggregates and ``operationalizes"
distinct dimensions of variation and tests distinct causal mechanisms
to approach this broad thesis in the most grounded and tractable ways
possible.

As a result, the broad theoretical argument which motivates and unites
the individual studies is rarely obvious within any particular article.
The following section, therefore, provides a stylized and somewhat
sociological narrative of the literatures on media and globalisation
in modern social science, highlighting the gaps which this dissertation
seeks to fill. In doing so, I also gesture to some broad historical
observations on the social sciences more generally insofar as it contextualizes
and foreshadows my arguments, given that contemporary social science
bears some political traces of both globalization and the transformation
of media environments since the 1960s.


\section{The Lost Tradition of Media as Propaganda}

From the 1920s until the end of World War II, the conventional wisdom
was that the role of mass media in modern society was, and ought to
be, an instrument of propaganda for the optimal functioning of the
state (\citealt{Bernays:2004vo, lippmann1932public}).
With the rise of information theory, which would become a basis for
modern computing (\citealt{Shannon:2013iy}; \citealt{gleick2011information}),
in the post-war period there was a flowering of social-scientific
efforts to link the propaganda-role of media to the burgeoning framework
of information theory (\citealt{wiener1965cybernetics, Deutsch:1953ww, Deutsch:1966ux, McLuhan:1994tf, Ellul:1965uf}).\footnote{To say nothing of concurrent and parallel movements in radical, continental theory. See \citet{Horkheimer:2009te}, \citet{adorno2001culture},
\citet{Debord:1967vn}. %
}

Today, these incipient social-scientific theories of the media appear
remarkably cynical: given the longstanding conventional wisdom of
elites that the media were mere tools of propaganda, the emergence
of legitimately scientific models of information quite naturally led
social theorists to conceptualize the media as instruments of social
control. Thus, these early efforts are laden with surprisingly sinister
vocabularies, the most recurring themes revolving around the control,
monitoring, and shaping of mass publics. 

These early social scientists of the media, most of whom were writing
within democratic states, had surprisingly little to say about the
media having anything to do with the empowerment of the masses. This
is puzzling given that, today, scholars and schoolchildren alike are
most immediately inclined to think of the media as a ``watchdog"
over government, the main purpose of which is to ensure popular sovereignty
through the free flow of information. Today, even those social scientists
most critical of media effects are exceptions which prove the rule,
as they typically frame their findings as raising questions about
the media's well-known function as government watchdog.

If the earliest and most influential social-scientific models of the
media were so cynical, then why, when and, how did contemporary social
scientists develop such a sanguine view of the media? I do not pretend
to offer any definitive answer to such questions, as the present volume
is a social-scientific study of media effects in international political
economy, not an intellectual history or sociology of ideas. Yet, it
is necessary to float some short and provisional answers to these
questions because the studies which follow are motivated and informed
by certain intuitions and provisional hypotheses regarding this peculiar
puzzle in the history of the social sciences. And substantively, the
studies here do begin to fill the gap between the early cynicism of
media scholars and the sanguinity of contemporary social scientists.
However, as is necessary in social science, to make the general, overarching
hypotheses analytically tractable, each of the studies in this volume
have so narrowly operationalized them that they literally would be
invisible were I to not state them up front. Thus, to provide helpful
background but also to adequately situate the importance of this volume,
it is worthwhile to sketch some of the more general and ambitious
hypotheses for which this volume does offer some provisional evidence,
however much it is only a beginning.

So where did our benign democratic notions of the media come from,
when for several decades the conventional wisdom on the role of media
was expressly anti-democratic and the very nature of information was
now becoming understood scientifically? This intersection in intellectual
history would seem to predict a future in which the various media
would become all the more powerful tools for small national elites
to control and manipulate mass publics, the vision largely shared
by so many incipient post-war social scientists. But yet, the notion
of propaganda recedes from the social sciences from its high point
around 1950, while the study of information continues increasing and
the social scientific study of media begins in earnest. 

In some sense, these social-scientific currents which are only just
beginning to theorize the mass media with an emphasis on propaganda
and information control are absorbed by government and the private
sector. It appears as if this incipient social-scientific perspective
is adopted and \emph{put into practice} by various branches of the
state, as in the rise of ``counterinsurgency'' abroad and government
``public relations'' at home, or otherwise the private commercial
development of communications engineering and ``operations management.''
As the new sciences of information control are put into practice by
the state and the private sector, it is at this time that the curiously
mild-mannered attitude toward the media is elevated into a baseline
for political science research (Lazarsfeld 1944; Berelson 1954; Campbell
et al. 1960).%
\footnote{The Columbia group around Lazarsfeld, from the beginning, was really
only interested in what was already a highly narrow and market-oriented
type of behavioral variation. Bartels notes how they only turned to
electoral behavior when they could not find grant money to study consumer
behavior (Rossi 1959, 15-16, as cited in Bartels 2008). The point
for our purposes is that these pioneering studies which would become
baselines for the modern study of political behavior rose to prominence
with a view of the media that already abstracted away from the more
``sinister'' media effects predicted by the group discussed above.
Thus, by the 1954 \emph{Voting: A Study of Opinion Formation in a
Presidential Campaign, }the Columbia group finds little evidence for
the role of parties and media in presidential campaigns. The later
Michigan group, whose election studies would become an increasingly
institutionalized backbone of American political science research,
also had a view of the media which is puzzlingly inane and conservative
read alongside work of roughly the same period by someone such as
Karl Deutsch. Of course, I do not here take issue with the validity
of these early findings as far as they go; my point is only to flag
that these foundational studies in American political science adopted
an approach which generated strikingly inane findings on the political
effects of media, especially when read alongside those who were grappling
with the more general historical functions of media as institutions
of social control.%
} This baseline conventional wisdom of ``limited effects'' from media
would no doubt be challenged within political science, but it nonetheless
has remained dominant (Katz; Bartels 2008). That research funding
distributed by the U.S. government and the private sector played a
prominent role in the mainstreaming and institutionalization of the
Columbia and Michigan models of political science research approaches,
at the very same time that information theory is being rapidly mobilized
in actual state and corporate operations and logistics, further tempts
one to the hypothesis that perhaps the greatest achievement of state
and corporate media control was to have ensured that social scientists
would never quite succeed in understanding or demonstrating the media's
function in social control.

This is why the present detour through intellectual history is not
merely an overly general literature review; rather, this somewhat
sociological review of the literature is itself suggestive empirical
evidence, however circumstantial, regarding a crucial transformation
in our thinking and practices of media politics. I have traced in
the record of the social sciences the transformation of media-as-propaganda
to media-as-transparency to outline a general gap in the literature
which this volume contributes to filling, but also to present some
provisional evidence, very close to home, of precisely how media-as-propaganda
may shape certain institutional political outcomes in ways which have
hardly been noticed. Indeed, if the media are most importantly propaganda
tools then, to the very degree they are politically effective, we
would expect them to go unnoticed by institutional social science.
Indeed, even the exceptions suggest evidence for this rule, for the
most popular intellectual inheritors of the media-as-propaganda tradition
today remain marginal to dominant mainstream social science.%
\footnote{For instance, see \citealt{Herman:1988ta,mcchesney2000rich,luhmann2000reality} %
}


\section{Globalization and its Variable Discontents}

These peculiar transformations in the study and practice of media
politics curiously precede the period of remarkable, worldwide economic
integration which has come to be known as ``globalisation."
Globalisation typically refers to the period from the early 1970s
to the present, when countries around the world saw large increases
in the flows of goods, services, and capital across borders. It is
widely thought by political economists that economic integration is
welfare-improving on net and in the long-run, yet even staunch ``free
market" economists acknowledge that economic integration
raises the incomes of certain domestic groups and lowers the incomes
of other groups, at least in the short-run. For this reason, globalisation
has brought with it many notable examples of political protest and
social unrest. Yet, discontent around economic globalisation has varied
across time and space and there remains much debate regarding the
conditions under which domestic political processes respond to economic
globalisation in different ways.

Both the concept and the processes of globalization have had a dubious
impact on the popular political imagination. The ideology which this
very concept bears witness to is that globalisation is a process,
a noun, something which has descended on the system of nation-states
from the outside, causing rather than being caused by the actions
of policymakers. As such, the very concept represents a political
bias because the casting of human actions as a process, the replacement
of verbs with nouns ending in ``-ation'', is a tendency higlighted
by critical discourse analysts of authority figures seeking to obscure
the reality of their power (Fowler 1979, 33-41).%
\footnote{Interestingly, it is also a tendency of social scientists (Billig
117).%
} It is well documented that politicians strategically deploy the rhetoric
of globalization to justify economic reform (\citealt{Hay:2011dh}),
and it has also been shown that the rise of economic globalization
weakens the tendency of voters to hold politicians accountable for
economic performance (\citealt{Hellwig:2007gn,Hellwig:2007jr}). 

Thus, the economics, rhetoric, and politics of economic globalisation
since the 1970s appear at first glance quite consistent with the economics,
rhetoric, and politics of the media at that time. If the disappearance
of propaganda theories during and after the war represented, as I
argued above, not the decline of that idea but rather its implementation,
then it would stand to reason that the role of media in promoting
state interests appears to have played a role in the popular and scholarly
narrative of economic globalisation since the 1970s. Specifically,
the overarching thesis of this volume--which remains too abstract
and provisional to permit rigorous direct testing, but which the present
studies begin to make tractable--is that the rise of modern media
around the world has, in different ways, helped state elites to promote
certain perceptions of international integration to fundamentally
undemocratic ends.


\section{Chapter Summaries}

This volume contains three independent efforts to operationalise this
stylised macro-historical hypothesis within a few different but well-established
literatures and theoretical frameworks in contemporary political science,
using quantitative as well as qualitative evidence.


\subsection{Mass Media and the Domestic Politics of Economic Globalization}

Some scholars argue that the spread of mass media around the world
will enable the political mobilization of previously disengaged domestic
groups: increased access to information and reduction of communication
costs should increase the ability of groups to mobilize around political
decisions that affect them. Others argue that mass media only empowers
the already empowered. In the latter case, the spread of mass media
should not have an egalitarian effect on distributive outcomes, and
could potentially increase pre-existing inequalities. I test these
competing conjectures in the context of the globalization-welfare
literature, asking whether mass media makes welfare-state effort more
or less responsive to the income losses of domestic groups facing
increased international exposure. I find that on the whole, in most
countries between 1960 and 2000, mass media depresses the positive
effect that globalization has on domestic welfare-state effort. To
supplement my interpretation of the cross-national findings, I also
employ individual-level survey data with unique measures of economic
blame attribution, including attribution of blame to international
forces. The survey data, from France in 1992 and 1993, shows that
mass media tends to diffuse perceptions of responsibility and in turn
shopes vote intentions accordingly.


\subsection{Why are the Most Trade-Open Countries More Likely to Repress the
Media?}

Why are more trade-open countries more likely to repress the media,
even though media freedom is positively correlated with most other
components of economic globalization? To explore and understand this
little-known empirical puzzle, I argue that economic globalization
exerts contradictory pressures on state-media relations. On the one
hand, economic openness encourages national policymakers to promote
media freedom because foreign investors are more likely to invest
where information is reliable. On the other hand, because adjusting
to economic openness implies distributive conflict which can threaten
the government, openness also generates incentives for national policymakers
to suppress information and communication about the costs of liberalization.
This paper develops a theoretical model that reconciles these contradictory
expectations by disaggregating economic globalization into its component
parts and distinguishing changes (liberalization) from levels of economic
globalization (openness). I argue that liberalization of trade, inward
foreign direct investment, and inward capital flows increase the probability
states will repress the media, as states seek to quell domestic conflict
around the adjustment costs of liberalization. In the long run, however,
different types of economic openness exert different pressures on
media freedom depending on how much they reward transparency. I argue
that financial openness leads to greater media freedom in the long
run because transparency is important to capital markets, but trade
openness exerts no positive effect on media freedom in the long run
because foreign importers and exporters are unaffected by transparency
in other countries. To test these expectations, I use a mixed-methods
research design employing large-N statistical tests combined with
process-tracing in Argentina and Mexico.

\subsection{Media Ownership and the Social Construction of Globalization}

If exposure to the international economy is something for which government
leaders have to compensate their constituencies, such international
forces have to be identified and explained to those who would suffer
from them. Knowledge of and opinions regarding the effects of globalization
may be determined by heuristics and cues from professional associations,
trade unions, and government leaders. But arguably it is the owners
and journalists of the mass media that are the most powerful set of
actors charged with identifying and explaining political forces not
directly observed by the public. Because the interests and incentives
of media owners are not necessarily consistent with the mass publics
they serve, I argue that the response of mass publics toward the global
economic exposure of their home country will vary according to the
different interests of different types of owners. The mechanism by
which this causal connection is likely to be realized is variance
in how globalization is represented in media reports. Different kinds
of media owners are biased by different incentives and are therefore
likely to represent globalization in observably different ways, ways
which are marginally more likely to produce mass attitudes consistent
with the owners\textquoteright{} interests.

I find mixed evidence, from three levels of analysis, that media ownership
significantly conditions the political response of mass publics toward
globalization. The main quantitative analysis reveals that the assumptions
of the compensation thesis are problematic: in relatively few of the
different model specifications examining different measures of globalization
and different atti- tudes toward government intervention was there
significant evidence that people demand government intervention to
compensate for exposure to global free trade. In relatively few cases
was the sign of the coefficient even as predicted by this thesis.
The main findings of interest, and the main potential contribution
of this paper, relate to the effect of media ownership in mediating
the political response to exposure to global free trade. Although
findings were not consistent and were very sensitive to model specification,
more than half of the total cross-national models showed that either
foreign or state ownership significantly dampened or reversed the
effect of some globalization process on a certain attitude.


%%%%%%%%%% CHAPTER BREAK %%%%%%%%%%%%%%%%%%%%%%%%%%%%%%%%%%%%%%%%%%%%%%%%%%%%%%%

\chapter{Why are the Most Trade-Open Countries More Likely to Repress
the Media?}

\section{Abstract}

Why are more trade-open countries more likely to repress
the media, even though media freedom is positively correlated with
most other components of economic globalization? To explore and understand
this little-known empirical puzzle, I argue that economic globalization
exerts contradictory pressures on state-media relations. On the one
hand, economic openness encourages national policymakers to promote
media freedom because foreign investors are more likely to invest
where information is reliable. On the other hand, because adjusting
to economic openness implies distributive conflict which can threaten
the government, openness also generates incentives for national policymakers
to suppress information and communication about the costs of liberalization.
This paper develops a theoretical model that reconciles these contradictory
expectations by disaggregating economic globalization into its component
parts and distinguishing changes (liberalization) from levels of economic
globalization (openness). I argue that liberalization of trade, inward
foreign direct investment, and inward capital flows increase the probability
states will repress the media, as states seek to quell domestic conflict
around the adjustment costs of liberalization. In the long run, however,
different types of economic openness exert different pressures on
media freedom depending on how much they reward transparency. I argue
that financial openness leads to greater media freedom in the long
run because transparency is important to capital markets, but trade
openness exerts no positive effect on media freedom in the long run
because foreign importers and exporters are unaffected by transparency
in other countries. To test these expectations, I use a mixed-methods
research design employing large-N statistical tests combined with
process-tracing in Argentina and Mexico.

\section{Introduction}

Given the conventional wisdom that democratic political institutions
drive economic openness (\citealt{Milner:2005ci}) and vice-versa
(\citealt{EICHENGREEN:2008gg}), it is surprising that since the 1960s,
on average, those countries which have been more open to international
trade have had lower levels of media freedom. Although international
portfolio capital and foreign direct investment are each positively
correlated with media freedom around the world, the bivariate relationship
between trade and media freedom is slightly negative.%
\footnote{Disaggregated economic data come from the World Bank Development Indicators (\citealt{WorldDevelopmentIn:2012wl}) and data on media freedom come
from Freedom House () and Van Belle's Global Press Freedom Dataset
(\citealt{van2000press}). See the section on Data and Method below
for a more detailed discussion of data and coding.%
} Considering the 151 countries between 1960 and 2011 for which there
is available data, those countries which most often had a repressive
media environment had higher levels of trade than those countries
which most often had a free media. This is true in democratic and
non-democratic countries, although the negative relationship is weaker
in democratic countries. Given the positive correlation found between
media freedom and other measures of economic openness such as portfolio
capital flows foreign direct investment, and the KOF Globalization
index (\citealt{dreher2008measuring}), the coincidence of high trade
openness and media repression is a surprisingly under-reported empirical
puzzle in international and comparative political economy.

This puzzle points to a larger gap in research on the domestic effects
of economic globalization. International and comparative political
economists have not yet developed a serious theoretical and empirical
account of how a country's media are likely to be affected by that
country's integration into the international economy. Much is known
about the effects of economic integration on aspects of domestic politics
such as cleavages (\citealt{Rogowski:1987ip}, \citeyear{Rogowski:1989wm};
\citealt{hiscox2002international}), growth rates (\citealt{Rodriguez:2001uw});
domestic spending (\citealt{Rodrik:1998te,Burgoon:2001dp}), civil
war (\citealt{Barbieri:2005uk,Bussmann:2007vx}), and generic measures
of democracy (\citealt{EICHENGREEN:2008gg,Li:2003vj}), but very little
is known about how economic integration affects state-media relations.
One exception is a working paper by Orion Lewis (\citeyear{Anonymous:lbhrCJXF}),
which finds mixed but suggestive evidence that trade openness is negatively
related to media freedom and portfolio capital is positively related
to media freedom. Other research has considered whether political
and civil liberties (broadly including freedom of the media) affect
international economic flows (\cite{Adam:2007gn}) and the effect
of media in economic reform (\citealt{Coyne:2004bq,Islam:2002uc}),
but in the extant literature there is no systematic analysis of whether
and in what ways domestic media freedom has been shaped by increasing
international economic integration around the world.

\begin{figure}
\begin{centering}
\includegraphics[scale=0.2]{article2_introplot.png}
\par\end{centering}
\caption{Mean Trade Levels and Mean Media Freedom, 1960-2011}
\end{figure}

The present study provides a theoretical account of how different
international economic flows affect domestic media freedom differently,
focusing on the puzzling negative correlation between trade levels
and media freedom. It improves on the limited previous research in
two ways. First, Lewis (\citeyear{Anonymous:lbhrCJXF}) uses only
the Freedom House measures of press freedom and therefore considers
country-level panel data only between 1993 and 2006. The present study
incorporates the Global Media Freedom Index by Van Belle (\citeyear{van2000press,Belle:1997wo}) to consider a similar panel of countries for the much longer timeframe from 1960 to 2011. Second, I emphasize the difference between levels
and changes (long-run and short-run effects) in a country's exposure
to international economic flows, whereas Lewis considers only levels. 

\section{Globalization, Democracy, and Media}

Previous research provides strong reasons to expect the global integration
of markets to exert pressures on institutions of democracy, but there
remains much theoretical uncertainty about the degree to which these
effects are positive or negative. Many have argued that economic globalization
generates economic growth which strengthens democratic institutions
(\citealt{Baghwati:1997vy}; \citealt{Im:1996cl}), increases incentives
for peace (\citealt{Baghwati:1997vy,Oneal:1999fc}), or diffuses democracy
as a norm (\citealt{Kant:1983uf,Limongi:1996dr}) On the other hand,
many have argued that economic globalization is negatively associated
with democracy because it rewards efficiency rather than popular sovereignty
(\citealt{Huntington:1975vt,Lindblom:1977ue,Cammack:1998gf}), or
because leaders may prefer to repress rather than compensate the domestic
losers from increased openness (\citealt{Adsera:2002vt}).\footnote{See \citealt{Li:2003vj} for a detailed overview of this debate.%
} Within the general debate surrounding the globalization-democracy
nexus, some researchers such as Li and Reveuny (\citeyear{Li:2003vj})
have sought greater clarity by disaggregating the distinct types of
international economic flows and considering them separately, but
relying on standard aggregate measures of democracy. Li and Reveuny
find that trade and portfolio capital have negative effects on democracy,
while foreign direct investment and democratic norms have a positive
effect, but their dependent variable of democracy is calculated with
the common procedure of subtracting the Polity autocracy score from
the Polity democracy score. Thus, despite much research on the relationship
between economic globalization and democracy, and despite evidence
that disaggregation is fruitful for understanding this nexus of relationships,
relatively little is known about how different international economic
flows affect the various institutions which separately constitute
what we know as democracy.

In particular, very little research to date queries whether and how
economic globalization shapes state policies regarding domestic media
freedom. One exception is Lewis (\citeyear{Anonymous:lbhrCJXF}),
who finds that FDI inflows are positively associated with press freedom,
trade levels are negatively associated with media freedom, and portfolio
capital inflows have no discernible effect on press freedom. But as
a first investigation into this question and as a largely inductive
effort to establish the statistical patterns, the theoretical interpretations
of this article are mostly provisional. Additionally, Lewis considers
only levels of trade and year-by-year inflows of FDI and portfolio
capital, whereas recent research shows that the distinction between
flows (year-to-year movements) and stocks (the sum of all previous
flows) is crucial in researching the effects of foreign investment
on repression (\citeyear{Sorens:2012hc}). As discussed above, Antonis
and Fillapaois (\citeyear{Adam:2007gn}) consider the effect of civil
liberties such as media freedom on FDI, but not whether FDI affects
civil liberties.

However, previous research on the relationship between markets and
media more generally provides a basis for theorizing the relationship
between economic integration and media freedom. Broadly, one tradition
argues that the spread of markets and freer media are positively associated
(\citealt{Habermas:1991vg,Islam:2002uc,Islam:2003tu}). However, an
opposite tradition suggets that markets and the logic of profits and
efficiency create incentives for authoritarianism (\citealt{Huntington:1975vt}).
With respect to media politics in particular, Gehlbach and Sonin (\citeyear{Gehlbach:2011ky})
show that larger advertising markets are associated with nationalization
of private media because, they argue, the benefits of state control
increase with the advertising market. If economic liberalization tends
to enlarge advertising markets by spurring economic growth, then liberalization
might increase the state's incentives to repress private media just
as it increases incentives to nationalize it. Furthermore, international
economic integration brings the threat of social and political backlashes
(\citealt{Bussmann:2007vx}), which require the state to compensate
the domestic losers from globalization (\citealt{Rodrik:1998te})
or, alternatively, repress them (\citealt{Adsera:2002vt}).

On the other hand, the literature on 'competitive authoritarianism'
suggests that increasing economic interdependence is one of the forces
which has increasingly rendered traditional authoritarian repression
unfeasible (\citealt[60, 62]{Levitsky:2002gx}). As a country becomes
increasingly integrated with the world economy, it increases the costs
of overt authoritarianism by increasing the salience of international
opinion, increasing the voice of domestic opposition, and increasing
the number of domestic actors affected by international perceptions
(\citealt{Levitsky:2006ex}). For example, Fujimori in Peru in 1992
and Putin in Russia in 1993 failed in their efforts to overtly circumvent
the legislature in part due to such international pressures (\citealt[56]{Levitsky:2002gx}).

International pressures against overt authoritarianism force regimes
to adopt formally democratic institutions such as elections, but often
leaves them free to violate human rights and civil liberties. For
example, in the US-Mexico negotiations leading up to the North-American
Free Trade Agreement, Mexican leaders made significant changes to
present a front of democracy and respect for human rights to encourage
investors, but there was no specific or formal conditionality which
would have prohibited or even discouraged the repression of civil
liberties if necessary.

At the same time, Levitsky and Way highlight the media as one of the
four main arenas in which incumbent governments can contest and subvert
international pressures to democratize. Competitive authoritarian
governments may permit a formally independent and relatively free
media, as in Peru, Serbia, Panama, or Nicaragua during the late 1980s
and much of the 1990s, while engaging in alternative, more subtle
tactics of repression, such as manipulative adminstrations of the
law or tax code (\citealt[53,58]{Levitsky:2002gx})

While economic integration engenders distributive conflicts which
tempt states to repress certain domestic groups at the same time it
disincentivizes certain overt techniques of repression, governments
around the world increasingly engage in strategic, authoritarian interventions
into domestic media politics. Corrales and Westhoff \citeyearpar{Corrales:2006vz}
find, for instance, that authoritarian regimes are more likely to
develop television than internet, because television is more easily
controlled. Additionally, many authoritarian regimes welcome the internet
but are actively pursuing techniques of information control and manipulation
\emph{on} the internet in a networked fashion (\citealt{MacKinnon:2011id,Pearce:2012fm}). These findings show that however much economic integration is making certain forms of repression obsolete, newer and more subtle techniques
of media repression remain both attractive and viable.

\section{Theory and Hypotheses}

As Antonis and Fillapaois point out, and as Lewis also argues, research
on the relationship between globalization and democratic institutions
likely shows such contradictory results because different international
economic flows exert different pressures. To build on this idea while
advancing the literature beyond the limitations discussed above, the
next section provides a more deductive account of preicisely why we
should expect various types of international economic openness to
exert different effects on media freedom.

Based on the review of previous research regarding the domestic political
effects of international economic flows and the media politics of
competitive authoritarianism, I develop a simple, informal rational-choice
model of how state media policy should respond to trade, foreign direct
investment, and portfolio capital flows. Consider a state which experiences
a variable increase in some inward, international economic flow of
trade, FDI, or portfolio capital. This increased flow will increase
the income of certain domestic groups and decrease the income of others,
according to well-developed open-economy expectations. The increased
economic flow can be thought of as random and exogenous or the result
of conscious state policy such as lowering tariffs. If they are well-informed
and mobilized, a domestic group which experiences a negative income
shock from economic liberalization would demand that the policymaker
either close the domestic economy or compensate the group for its
income loss, or else face rebellion. The ``rebellion'' could be
electoral if the state is a democracy or a violent insurgency if the
state does not have institutions to facilitate peaceful change. After
experiencing the international shock, the media, if free to do so,
would report the protests of the aggrieved group and its causes, namely,
increased national exposure to the international economy and its conflictual
distributive consequences. However, if the media does not fully report
the political context and consequences of the international shock,
the group which suffered an income loss would not threaten rebellion
at all. A free media, in other words, are essential for domestic losers
from globalization to exercise power in the domestic politics around
the distributive outcomes of economic globalization. Where there is
a free media, domestic losers from globalization hold policymakers
accountable, but where there the media are manipulated by government,
the claims of aggreived groups cannot exact concessions from holding
policymakers accountable. This can be from either not reporting and
therefore not informing domestic groups of the political-conflictual
nature of globalization, or from silencing those claims if and when
they are made.

After increased exposure occurs, the policymaker would prefer not
to compensate the domestic group but prefers compensation to facing
rebellion or closing the economy to \textit{ex ante} levels. The policymaker
can close the political process to any competitors to obviate the
political pressure to compensate them (\citealt{Adsera:2002vt}),
but the higher their level of integration, the more costly are overt
types of repression (\citealt{Levitsky:2002gx}). Supposing that a
policymaker can choose among compensating the aggrieved domestic group,
excluding competitors from the political process, or engaging in some
repressive practices which vary on a continuum from overt to covert.
In effect, we can conceptualize their utility function as including
a penalty on overtness which increases with the country's economic
integration, such that there is decreasing utility to overt forms
of repression such as outright exclusion from the political process
or government killings but this disutility approaches zero for repressive
tactics which are relatively obscure such as the selective prosecution
or financial targeting of opponents.

Among the less visible ways of exercising anti-democratic control,
information-communication control will be uniquely attractive for
the policymaker. This is because not only are repressive media tactics
less severe than mass killings or canceling elections, but because
control of the media could potentially tame international judgments
independently by shaping what gets reported internationally. That
is, control of the media can first minimize policymaker accountability
for the adjustment costs of liberalization by suppressing domestic
dissent, but policymakers could also reasonably expect that suppressing
information at home would decrease the flow of negative information
abroad, promoting their international image in part by repressively
shaping their image at home. In summary, increasing linkages to other
states and international pressures raise the cost of overt repression
for liberalizing states, which increases the attractiveness of more
subtle, lower-visibility tactics for suppressing dissent against liberalization.
Repression of the media stands out as a uniquely attractive first
because it is precisely such a relatively low-visibility, low-salience
type of repression but also because if successful it would tend to
lower negative visibility in general.

\subsection{Differences among trade, foreign direct investment,
and portfolio capital}

The previous subsection argues that media repression is uniquely attractive
to incumbents presiding over economic liberalization. However, the
international actors who are the counterparties to a country's international
exchanges are also strategic actors. When a government represses the
flow of information and communication within its territory, these
counterparties will respond strategically depending on how their particular
investment in the country is affected by domestic freedom of information
and communication. Given that these international counterparties have
very different stakes in domestic media freedom depending on whether
they are engaged in trade, foreign direct investment, or portfolio
capital investment, the utility of media repression during economic
liberalization will be conditioned according to a country's composition
of exposure to these flows.

\subsubsection{FDI}

FDI is defined as the private capital flows from one firm to an enterprise
located in a country outside of the firm's home nation. FDI flows
consist of equity capital, intercompany debt, and reinvested earnings,
whenever the investment is sufficient to give the firm a controlling
stake (typically 10\%) in the enterprise (\citealt[9]{DirectInvestmentTechnicalExpertGroupDITEG:2004wa}:
\citealt[588]{Jensen:2003to}) Foreign direct investment is unique
among other types of international investment in that FDI involves
a longer-term committment and thus the interests of FDI investors
are relatively more aligned with the long-term interests of host countries
(\citealt{Lipsey:1999tn}; \citealt[588]{Jensen:2003to}). The standard
economic theory of FDI suggests that firm-level investment decisions
to invest directly in a foreign country are not based on relative
factor endowments or comparative rates of return, but on domestic
market imperfections which can be exploited by multinational corporations
(MNCs) better than domestic firms (\citealt{Hymer:1960vo}; \citealt{dunning2013international}).
The distributive consequences of FDI inflows are complex: FDI is typically
thought to increase inequality between skilled and unskilled workers
as MNCs tend to be technologically skill-biased relative to domestic
firms (\citealt{Feenstra:1997kx}) and unskilled, subsistence farmers
do not have the resources to become entrepreneurs (\citealt{Basu:2007ir}).
However, FDI is also thought to decrease overall domestic income inequality
as an increase in the supply of capital relative to labor increases
wages and reduces inequality between capital and skilled labor (\citealt{Jensen:2007fr}).
Jensen and Rosas present evidence that, because poor countries have
relatively little skilled labor, FDI's effect on closing the gap between
skilled labor and capital is likely to decrease inequality on net
even if it increases inequality between skilled and unskilled labor.
Thus, FDI inflows generate distributive conflict among skilled and
unskilled labor, but are unlikely to generate highly salient distributive
conflict overall. This expectation is borne out by research on the
relationship between economic globalization and civil war. Bussmann
and Schneider (\citeyear{Bussmann:2007vx}) find, contrary to their
expectations, that inflows of FDI decrease rather than increase the
likelihood of civil war onset.

Most significantly, of the three types of international economic actors
considered here, investors of FDI have a long-term stake in the conditions
of a host country. Because of this, despite long-standing expectations
that foreign direct investors prefer the efficiency of authoritarian
regimes, the balance of evidence suggests that democracies draw greater
FDI flows than autocracies because they are more credible (\citealt[588]{Jensen:2003to}).
Some scholars have sought to extend this logic by arguing that FDI
should be attracted to respect for human rights (\citealt{Blanton:2007ep})
have faced problems of measurement and missing data (\citealt{Sorens:2012hc}).
After accounting for these issues, Sorens and Ruger find no link between
FDI and human rights. Thus, while formal democracy attracts FDI and
FDI does not appear to generate intense distributive conflicts, neither
does it appear to ``punish'' governments for violating human rights.

Interestingly, Antonis and Filipaios find, consistent with Jensen,
that FDI seeks strong political rights while its attraction to civil
rights is hump-shaped such that FDI is associated with both high and
low levels of civil rights (\citeyear{Adam:2007gn}). One possible
explanation of these inconsistencies is that the socially positive
consequences of FDI (rewarding democracy and rule of law and decreasing
civil war onset) occur at the same time as, or perhaps in part through,
the repression of civil rights. This is consistent with the model
presented above, wherein the repression of a particular civil right
(the freedom of expression) embodied in media freedom is repressed
to dampen the disruptive effects of new FDI inflows, but in the long
run equilibriates to a high level alongside political rights.

Thus, as argued in the previous section, increases in FDI should be
associated with media repression, but in the long run FDI should be
associated with media freedom. Because FDI is not averse to violations
of rights per se, and is perhaps attracted to governments with low
respect for civil rights, governments will use media repression to
suppress distributive conflicts associated with FDI, but after the
adjustment takes place and threat of conflict subsides, then FDI in
the long-run should be associated with media freedom for the same
reasons it is associated with democracy, namely credibility and stability.

\subsubsection{Portfolio capital}

Portfolio capital is defined as the purchase of stocks and bonds of
less than 10\% of the outstanding stock of foreign firms (Kenen 1994,
Walther 1997). The standard economic theory is that portfolio capital
tends to flow where the rate of return on the target country's domestic
assets is high relative to the riskiness of the investment (\citealt[743]{mosley2003global};
\citealt[685]{ISQU:ISQU420}). Portfolio capital is distinguished
by its short-term, speculative nature compared to FDI. In a benchmark
study of how portfolio investors evaluate political risks, Bernhard
and Leblang (\citeyear{Bernhard:2002gy}) show that portfolio investors
respond to changes in country's political system (such as elections),
but not to the substance of those changes (for instance, partisanship).
Brooks and Mosley (\citeyear{Brooks:2007we}) show that portfolio
investors do respond to the substance of policymaking, such as partisanship
and macroeconomic priorities, but only in low-information environments
such as electoral turnovers. The effects of partisanship and macroeconomic
policy on portfolio capital decrease when the political system itself
is stable. The overall point is that portfolio investors are first
and foremost interested in stability and predictability rather than
particular policies, which only matter in periods when the predictability
of the future is low.

Portfolio capital inflows tend to appreciate the domestic currency,
which makes imports relatively cheaper in the home market and exports
relatively more expensive to foreigners. This will harm exports, leading
possibly to unemployment or decreases in wage levels in export-intensive
industries. It will also make it harder for domestic firms to compete
with relatively cheaper imports, also possibly leading to unemployment
or wage decreases. Finally, cheaper capital imports can encourage
skill-biased shifts in technology usage, increasing the incomes of
skilled labor and decreasing the incomes of unskilled labor (\citealt{Cragg:1996iy};
\citealt{Ros:2000vy}). Finally, because portfolio capital is relatively
liquid, the threat of sudden withdrawal by international investors
is well-known to have highly negative macroeconomic effects, such
as in Mexico in 1995 and Argentina in 2001.

Given the interests of governments and portfolio investors, governments
should be inclined to repress the media in response to the distributive
effects of portfolio capital for two reasons. First, inflows of portfolio
capital will make governments more beholden to the prevention of systemic
political risks such as general strikes, expropriations, or revolutions
(\citealt{Clark:1997jg}). This is consistent with anecdotal evidence
of portfolio investors who prefer governments to repress social unrest.
Neoliberal economic reforms including international liberalization
are often followed by large increases in foreign portfolio capital,
and there is anecdotal evidence that in some cases foreign investors
demand repression explicitly, such as when Chase Bank's Emerging Markets
Group circulated a memo urging Mexican President Ernesto Zedillo to
``eliminate the Zapatistas'' and their uprising in Chiapas in 1994
(\citealt{Silverstein:1995wc}). Second, given that policy is evaluated
by foreign investors largely in light of what is already known about
the government and its history, incumbents who preside over financial
liberalization for that very reason are likely to be sufficiently
trusted by foreign capital that relatively subtle tactics such as
media repression would be unlikely to shake confidence, especially
if it is in the interest of preventing larger disruptions such as
rebellions. It may be objected that portfolio investors would dislike
media repression because they rely on a reliable flow of information
regarding the country's conditions, but through modern ``news management''
politicians can practice a highly nuanced kind of transparency for
international observers and also seek to repress domestic media using
underhanded tactics. Indeed, country's which are open enough to receive
capital inflows are likely to already be relatively transparent in
the ways most relevant to investors, and this transparency required
to induce investment might even embolden the assertiveness of domestic
media.%
\footnote{This appeared to happen in Mexico during the 1980s and 90s (\citealt{lawson2002building})%
} Portfolio investors can typically rely on international news sources
which are less likely to be targeted within the host country (on account
of their financial independence and being linked to another sovereign,
such as that one in Argentina). Finally, portfolio investors often
have access to private, elite channels which provide them with politically
important information about foreign country conditions before it would
even be reported by free media (\citet{Dube:2011uv}).

Thus, we should expect media policy to respond to inflows of portfolio
capital just as it responds to FDI inflows. As a country adjusts to
the destabilizing distributive effects of international portfolio
investment, governments will be more likely to repress the media as
a relatively discreet tactic of pacifying social unrest, consistent
with investors' interests in stability. However, inflows of portfolio
capital will in the long-run be associated with media freedom, as
portfolio investors prefer high-information environments \emph{ceteris
paribus}.

\subsubsection{Trade}

Trade, defined simply as imports plus exports as the percentage of
a country's gross domestic product, is unique among the previous two
components of economic globalization in that the international counterparties
have no direct economic stake in the social and political conditions
of the home country. Put simply, trade is not an investment as are
FDI and portfolio capital flows. The standard economic intuition explaining
trade flows, although many sophisticated variations and extensions
have been developed, is still the well-known Ricardian theory of comparative
advantage. Other things equal, countries will tend to specialize in
producing for export those goods which they are most advantaged in
producing, and import from foreign producers those goods which domestic
producers are unable to produce as efficiently.

International trade theory and much research in political science
provides well-established expectations regarding the distributive
effects of a country increasing its exposure to international trade.
The standard Stolper-Samuelson model (\citeyear{Stolper:1941vp})
expects that increasing trade openness increases the income of the
domestically abundant factor while decreasing the income of the domestically
scarce factor. Thus, in capital-rich countries (industrialized or
post-industrial countries), increasing trade openness benefits capital
and harms labor, whereas in capital-poor countries increasing trade
openness is expected to benefit labor and harm capital owners. In
his benchmark study on the political consequences of these distributive
expectations, Rogowski (\citeyear{Rogowski:1989wm}) finds strong
evidence that domestic political coalitions are empowered and disempowered
by international trade as the Stolper-Samuelson model predicts. Hiscox
(\citeyear{hiscox2002international}) further refines these expectations
by showing that history is more finely explained by distinguishing
the relative mobility of factors: when domestic factors are relatively
immobile within the domestic economy, we do not observe class-based
cleavages but rather sector-based cleavages and cross-class alliances,
as immobility weds the interests of labor and capital to their shared
industry. In turn, the threat of distributive conflict from international
trade has been found salient enough to explain domestic political
outcomes as diverse as the size of welfare states (\citealt{Cameron:1978vb};
\citealt{Burgoon:2001dp}) and the onset of civil wars (\citealt{Bussmann:2007vx}).

The international counterparties to a country's international trade
have a uniquely low stake in the political stability of the country,
for the simple reason that the import and export of goods and services
is not directly affected by the sanctity of civil rights such as freedom
of expression or media freedom. Although emerging international norms
of ``corporate responsibility'' and ``fair trade'' are increasingly
visible in marketing for consumers in the wealthy democracies, these
norms revolve around specific labor market issues such as child labor,
``sweatshops'', and wages paid to workers in developing countries
(\citealt{Moore:2004gy}). Even if some consumers in the wealthy democracies
are increasingly willing to pay for more humane production conditions
in foreign countries (effectively an international tax on repressive
production conditions), there is no evidence and little reason to
believe that economic behavior in importing or exporting goods and
services anywhere in the world is in any way responsive to the sanctity
of significantly less salient civil rights such as media freedom.
For instance, consumers in the global North may very well prefer to
pay premiums for coffee explicitly labeled as ``fair trade,'' but
this provides no reason to expect they would pay more or less depending
on whether the exporting country's trade agreements were facilitated
by media repression. Similarly, if exporters in one country benefit
from lowered tarrifs in a foreign country, compared to FDI and portfolio
investors, they have uniquely less at stake in the political consequences
faced by the foreign country with rising imports.

Thus, as with FDI and portfolio capital, we expect changes in trade
openness to be associated with a higher probability of media repression,
but unlike FDI and portfolio capital, we expect this effect of trade
liberalization to persist into the long run, as governments face no
pressure from their counterparties to eventually transition to a free
media environment. If true, this would explain the puzzlingly negative
correlation between international trade and media freedom despite
the positive association between trade and most other components of
economic globalization.

\subsection{Hypotheses}

To summarize, the hypotheses of the preceding subsections can be stated
concisely as follows.

H1: Levels of trade openness decrease media freedom.

H2. Levels of portfolio capital increase media freedom.

H3. Levels of foreign direct investment increase media freedom.

H4. Changes in trade, equity, and FDI decrease media freedom.

\section{Data and Method}

To assess the theory, this article pursues a mixed-method research
design employing large-N statistical tests and qualitative within-case
analysis on two historically important cases. The intuition behind
this research strategy is that statistical analyses are necessary
to disentangle the independent effects of each economic flow, especially
in distinguishing between short-run and long-run effects, while qualitative
analysis is necessary for establishing the existence of a causal process.

In the quantitative analyses, I use state-level economic data from
the World Bank (\citeyear{WorldDevelopmentIn:2012wl}) for the main
independent variables of interest (FDI, portfolio capital, and trade)
for all available countries between 1960 and 2010. For the dependent
variable, I use the well-known Freedom of the Press scores from Freedom
House (CITATION) as well as the the Global Media Freedom Database
Van Belle (\citeyear{Belle:1997wo,van2000press}). Freedom House measures
press freedom on a continuous scale from 0 to 100 and covers most
countries from 1994 to the end of the economic time-series, whereas
Van Belle's data is essentially a dichotomous measure of media freedom
(footnote: it reduces to that) and covers most countries from 1948
to 1995. To maximize the sample, I create a single measure of press
freedom which converts the Freedom House scores for 1994-2010 to Van
Belle's dichotomous scale by interpolating from the two years for
which there are observations on both measures.%
\footnote{Specifically, I fit a logistic regression of the dichotomous variable
on the continuous variable for the three years of overlapping observations,
1994-1996. Then for each year post-1995, I imputed to each each country
a 1 on the dichotomous scale for each year their value on the continuous
scale had greater than or equal to a .5 probability of being classified
by the model as 1 (roughly all values greater than about 61 on the
100-point scale), and I gave them a 0 otherwise.%
} This operation is somewhat inefficient (information is lost from
the continuous measure) and biased (for some countries it generates
artificial ``changes'' in value between the two time periods) but
it credibly and consistently extends the time series for most countries
and represents the most spatially and temporally extensive single
measure of media freedom available. Thus, I employ it for the primary
analyses but then model and account for possible spurious ``effects''
due to this operation. To distinguish the separate effects of different
economic flows on media freedom across countries and over time, I
use a series of cross-sectional time-series regression techniques
and robustness checks, including panel vectorautoregression to check
against reverse causality.

To corroborate the quantitative findings and enhance our understanding
of the key puzzle motivating this paper, a following section offers
two within-case analyses which trace the process whereby trade liberalization
exerts pressure on domestic media freedom. To help control for confounding
spatial and temporal factors, I consider two ``third-wave''
democracies from the same region in the same time period, Argentina
and Mexico in the period between 1990 and 2011. These countries are
analytically well-suited for further examination because they both
democratized beginning in the 1980s and were consolidating in the
1990s. In autocratic regimes, even if we observed instances where
media repression follows economic liberalization, it would be hard
to infer that liberalization caused media repression because the media
repression could be a function of auotcracy in general. On the other
hand, if media repression follows economic liberalization in countries
which are otherwise politically liberalizing, it will be more credible
to infer that possibly economic liberalization generated the tendency
to media repression. Indeed, Argentina and Mexico are least likely
cases to expect media repression at this time because Argentina's
Carlos Menem and Mexico's Carlos Salinas were championed by American
politicians as models of democratic economic liberalization. Latin
America is also a substantively attractive region for further study
because Latin America is typically considered the first region where
democracies were able to implement politically difficult ``stabilization''
policies. In the 1970s, it was a puzzle how economic liberalization
would ever be achieved in democratic settings, given the status quo
bias of elected politicians and the popular support for protectionist
policies. An implication of this paper's argument, however, is that
even in formally democratic countries economic liberalization may
in some cases induce anti-democratic tactics such as media repression.
If this is argument is correct, then substantively it would be most
rewarding to better understand these cases which the conventional
wisdom holds to be democratic sucess stories. Finally, Mexico, unlike
Argentina and many other Latin American countries, did not experience
a deeply repressive military junta in the twentieth century. Thus,
if it is plausible that a government's historical legacy of repression
could alone make media repression in a later period more or less likely,
then we can be confident this is not an unobserved variable generating
outcomes in both Mexico and Argentina.

Specifically, I offer two short, ``disciplined-configurative'' case
studies for the purpose of better understanding these historically
important cases and to further test for the presence of a causal process
(\citealt[75]{george2005case}). I use a combination of structured,
focused comparison and process-tracing, asking specific questions
about the hypothesized process in each case and weighing the empirical
results against what the theory expects. Specifically, I ask the following
three questions. What was the policy background as well as the magnitude
and timing of trade exposure? What was the magnitude and timing, if
any, of social unrest and was it observably in response to the distributive
effects of trade? What was the magnitude and timing, if any, of government
efforts to restrict freedom of the media? After investigating the
historical record, I outline the answers to these questions and discuss
how well they fit the theoretical model.

\section{Analysis}

Table 1 displays the results of 5 different logistic regressions assessing
the probability of observing media freedom in countryit.
The model results provide fairly strong evidence that levels of trade
and portfolio capital inflow have opposite long-run effects on media
freedom as predicted in Hypotheses 1 and 2. The results provide no
evidence for any relationship between foreign direct investment and
media freedom as predicted in Hypothesis 3, however. Finally, as expected
by Hypothesis 4, the data reveal negative correlations between economic
liberalization and media repression for each type of openness, although
none of these estimates are reliable at any greater than an 80\% confidence
level (in the case of trade liberalization).

Model 1 is a baseline model which predicts media freedom simply as
a function of democracy, GDP per capita, and lagged levels of media
freedom. The model correctly predicts the state of media freedom in
most country-years. Model 1 correctly predicts 98\% of country-years
in which media are repressed (incorrectly classifying 47 cases out
of 2,027) and 96\% of cases in which the media are free (incorrectly
classifying 57 out of 1,432). Considering the logit estimates in terms
of probability rather than log-odds, ceteris paribus, for a country
which shifts from full autocracy to full democracy the probability
of observing media freedom is expected to increase 33\% from .11 to
.44. The effect of a two-standard-deviation increase in GDP per capita
(from a mean of \$6,359 to \$24,097), ceteris paribus, increases the
probability of observing media freedom 34\% (from .33 to .67).

Model 2 adds the openness variables to Model 1. Both trade and portfolio
capital are signed as expected and significant at the 95\% level,
consistent with the expectations of Hypotheses 1 and 2. FDI has no
statistically discernable effect. Although the model furnishes some
evidence that higher trade levels are associated with media repression
and higher portfolio capital flows are associated with media freedom,
Model 2 classifies the observed outcomes exactly as well as Model
1. Thus, while Model 2 is meaningful as an initial test of the theoretical
claims outlined above, it does not substantively improve our ability
to predict variation in media freedom.

Comparison of Model 3 and Model 4 represents a similar exercise as
in Models 1 and 2, but using an error-correction specification (levels
and changes of all independent variables) with natural cubic splines
of time instead of a lagged dependent variable. (\citealt{Beck:1998wg}).%
\footnote{Because BTSCS data are essentially grouped duration data, a lagged
dependent variable on the right-hand side of the equation has a much
more ambiguous meaning than it does for a continuous outcome variable.
Following Beck 2011 289, if a state has a high propensity to free
(or repress) the media at t-1 but does not, it is unclear whether
this should be interpreted as increasing or decreasing its propensity
at time t. Models 1 and 2 include a lagged dependent variable as a
harder test for the possibility that the coefficients of interest
are driven by duration dependence, but the prevailing best practice
is the use of splines as suggested by Beck et al. Finally, my efforts
to explore a lagged dependent variable in Models 3 and 4 was barred
by overfitting.%
} Model 4 provides additional support for the theory that trade levels
exert a negative effect, and portfolio capital a positive effect,
on media freedom. Both are statistically significant at the 90\% level,
controlling for both levels and changes in democracy and GDP per capita.
As in Models 1 and 2, accounting for openness does not strongly alter
our ability to predict outcomes in the sample. Whereas Model 3 incorrectly
classifies 203 country-years of media repression, Model 4 improves
this classifcation by incorrectly specifying 195 cases. On the other
hand, Model 3 incorrectly classifies 215 cases of media freedom and
Model 4 incorrectly specifies 230. However, the signficance of the
openness coefficients suggests that the relatively similar substantive
predictions from Models 3 and 4 reflect that historically media freedom
is perhaps overdetermined by openness, democracy, and economic growth.
In other words, although Models 3 and 4 do not offer many different
predictions in particular cases, the results nonetheless suggest that
any historical interpretation of media freedom which sees it as merely
a function of democratization or modernization would be imputing to
these factors causal effects some portion of which is possibly due
to international economic openness.

Model 4 also presents suggestive but inconclusive evidence for Hypothesis
4 regarding the negative effect of economic liberalization. The negative
coefficients for each differenced variable are consistent with the
expectation that all types of economic liberalization induce media
repression in the short-run, although none of these estimates are
statistically significant. However, if all types of liberalization
induce media repression and all types of liberalization are correlated
with each other, then the data could be insufficient for the model
to parse even real indepedent effects. Model 5, then, replaces the
liberalization variables with the simple sum of the three. As expected,
the standard error is significantly smaller, statistically significant
at an 88\% confidence level. Although the standard error is larger
than the conventional cut-off for statistical significance (and other
resulst reported above are similarly around the 90\% confidence level,
less than the conventionally prefered 95\%), the difference between
95\% and 88\% is itself of dubious significance. In my own view, given
that the research design includes a qualitative component which will
offer independent and alternative evidentiary weight, confidence levels
near 90\% are high enough that we should not yet fail reject the null
hypothesis.

\subsection{Effect Sizes}

Based on Model 5, the probability of observing media freedom in a
country completely closed to international trade is .48 holding all
the other variables at their means. Moving to the mean level of international
trade in the sample (71\% of GDP) decreases the expected probability
of media freedom by .9. The probability of observing media freedom
in the most trade-open countries (greater than 200\% of GDP) is roughly
23\% less probable than in a trade-closed economy.

The probability of observing media freedom in a country completely
closed to portfolio capital is .36 holding all the other variables
at their means. Moving to the mean level of portfolio capital in the
sample (.05\% of GDP) increases the expected probability of media
freedom by only .03. But the probability of observing media freedom
in the countries most open to portfolio capital (greater than 5\%
of GDP) is roughly 60\% more probable than in a hypothetical country
completely closed to portfolio capital.

\begin{centering}

\pagebreak{}
\begin{table}
\begin{tabular}{lD{.}{.}{3}@{\hspace{2em}}D{.}{.}{3}@{\hspace{2em}}D{.}{.}{3}@{\hspace{2em}}D{.}{.}{3}@{\hspace{2em}}D{.}{.}{3}} \toprule 
 &  \multicolumn{ 1 }{ c }{ (1)Baseline LDV } & \multicolumn{ 1 }{ c }{ (2)Openness+LDV } & \multicolumn{ 1 }{ c }{ (3)Baseline Spline } & \multicolumn{ 1 }{ c }{ (4)Openness+Spline } & \multicolumn{ 1 }{ c }{ (5)$X_{it}$ + $\Delta X_{it}$ } \\ \midrule
 (Intercept) & -3.691 ^*   & -3.287 ^*   & 41.520      & -71.072     & -48.327    \\ 
            & ( 0.129 )   & ( 0.148 )   & ( 42.089 )  & ( 109.662 ) & ( 125.215 )\\ 
lpolity2    & 1.800 ^*    & 1.726 ^*    & 4.381 ^*    & 4.372 ^*    & 4.436 ^*   \\ 
            & ( 0.207 )   & ( 0.256 )   & ( 0.484 )   & ( 0.508 )   & ( 0.524 )  \\ 
lgdpcap     & 1.066 ^*    & 1.624 ^*    & 1.847 ^*    & 2.594 ^*    & 2.491 ^*   \\ 
            & ( 0.146 )   & ( 0.225 )   & ( 0.429 )   & ( 0.413 )   & ( 0.418 )  \\ 
lfp         & 6.355 ^*    & 5.868 ^*    &             &             &            \\ 
            & ( 0.209 )   & ( 0.237 )   &             &             &            \\ 
ltrade      &             & -0.520 ^*   &             & -0.466 ^+ & -0.468 ^+    \\ 
            &             & ( 0.179 )   &             & ( 0.270 )   & ( 0.283 )  \\ 
lfdi.in     &             & -0.262      &             & 0.179       & 0.131      \\ 
            &             & ( 0.220 )   &             & ( 0.162 )   & ( 0.221 )  \\ 
lportfolio  &             & 2.191 ^*    &             & 1.533       & 2.203 ^+     \\ 
            &             & ( 0.771 )   &             & ( 0.991 )   & ( 1.334 )  \\ 
splinedf    &             &             & -0.021      & 0.036       & 0.024      \\ 
            &             &             & ( 0.021 )   & ( 0.055 )   & ( 0.063 )  \\ 
splinedf'   &             &             & -0.034      & -0.082 ^+   & -0.074     \\ 
            &             &             & ( 0.023 )   & ( 0.044 )   & ( 0.049 )  \\ 
trade.d     &             &             &             &             & -0.121     \\ 
            &             &             &             &             & ( 0.093 )  \\ 
fdi.in.d    &             &             &             &             & -0.050     \\ 
            &             &             &             &             & ( 0.117 )  \\ 
portfolio.d &             &             &             &             & -0.428     \\ 
            &             &             &             &             & ( 0.438 )  \\ 
polity2.d   &             &             &             &             & -0.135     \\ 
            &             &             &             &             & ( 0.118 )   \\ \midrule 
 $N$   & \multicolumn{1}{c}{5491    } & \multicolumn{1}{c}{3564    } & \multicolumn{1}{c}{5530    } & \multicolumn{1}{c}{3580    } & \multicolumn{1}{c}{3416    }\\ 
$AIC$ & \multicolumn{1}{c}{1131.870} & \multicolumn{1}{c}{851.578 } & \multicolumn{1}{c}{3658.104} & \multicolumn{1}{c}{2338.076} & \multicolumn{1}{c}{2241.897} \\ \bottomrule  
\end{tabular}

\caption{Determinants of Media Freedom, BTSCS regressions}
\end{table}

\pagebreak{}
\begin{figure}
\includegraphics[scale=0.2]{article2_trade_effect_plot.png}
\caption{Predicted effect of trade on media freedom (1000 simulations with
GAM smooth line)}
\end{figure}

\pagebreak{}
\begin{figure}
\includegraphics[scale=0.2]{article2_portfolio_effect_plot.png}
\caption{Predicted effect of inward portfolio capital level on media freedom
(1000 simulations with GAM smooth line)}
\end{figure}

\pagebreak{}
\begin{figure}
\includegraphics[scale=0.2]{article2_liberalization_effect_plot.png}
\caption{Predicted effect of year-to-year change in additive index of openness}
\end{figure}

\end{centering}


\subsection{Checking Reverse Causality with Panel VAR}

If already repressed media environments are more likely to open their
economy, then it is possible that our interpretation of the data wrongly
specifies the direction of the causal effect. To test whether trade
openness tends to precede media repression or media repression tends
to precede trade openness, I use a method of vectorautoregression
for panel data. Here I use only the continuous numerical measure of
media freedom supplied by Freedom House, beginning in 1994. The
panel vector-autoregressions suggests that shocks to trade affect
press freedom, controlling for democracy and GDP per capita, although
the effects decay quickly. There is no evidence that press freedom
affects trade.

\begin{centering}
\begin{figure}
\includegraphics[scale=0.5]{article2_var_trade.png}
\caption{Impulse response of press freedom to a one standard deviation shock
to trade level at t-2, with democracy and GDP per capita endogenous;
Green and yellow lines represent 95\% confidence intervals drawn using
Monte Carlo simulation (500 repetitions)}
\end{figure}
\end{centering}


\subsection{Within-Case Analysis}

In this section, I turn to the brief case studies of Argentina and
Mexico to assess qualitatively whether we can observe implications
of trade liberalization generating pressures toward media repression.

\begin{centering}

\begin{figure}
\includegraphics[scale=0.2]{article2_regions.png}
\caption{Economic Openness and Press Freedom 1960-2011, By Region}
\end{figure}


\begin{figure}
\includegraphics[scale=0.2]{article2_AM_cases.png}
\caption{Levels of openness and press freedom in four countries, 1990-2011
(standardized within countries)}
\end{figure}

\end{centering}

\subsection{Argentina}

Immediately upon inauguration as President in 1989, Carlos Menem announced
a package of neoliberal economic reforms which include liberalization
of international trade and capital flows, as well as privatizations
and spending cuts. (\citealt[189]{Tommasi:1995wx}; \citealt{Borner:2002cp}).
In the beginning of 1989, the average tariff rate is 39\% (a maximum
import tariff was 50\% with a tariff surcharge of 15\% on all imports).
By the end of 1989, the maximum tariff is 35\% and the average tariff
rate falls to 12\%. In 1990, all import licensing requirements are
abolished and tariffs are reduced across-the-board to 21\%. By 1995,
the average unweighted tariff is 10.5\% and non-tariff barriers as
well as export restrictions are removed (\citealt[7]{Beker:2011vq}).%
\footnote{Exemptions were made for IT, domestic appliances, and autos. As a
result, imports increased from \$4.1 billion in 1990 to \$21.6 billion
in 1994, while exports increased from \$3.7 billion to \$20.1 billion
at the same time. See \citet{Beker:2011vq}.%
}

As import competition put pressure on previously protected firms,
about 30\% of manufacturing employment was destroyed between 1992
and 1996. In those industries where import penetration increased the
most, wage inequality also widened during this period. Argentina's
Gini coefficient for income inequality, one of the lowest in Latin
America at the time, increased from 40.0 in 1991 to 47.4 by 1998.
(\citealt[505]{Galiani:2003fr}; \citealt[11]{Beker:2011vq})

Argentina will join the Mercosur customs union in 1991 and sign a
trade agreement with the United States in 1994. The reform package
largely succeeded in taming inflation rates and growing the economy.
The inflation rate shrinks from 5\% in 1989 to .16\% by 1996 and gross
domestic product grows by 40\% between 1990 and 1994 (\citealt[4]{Beker:2011vq}).
The IMF, World Bank, and the US government saw Argentina as a model
student in this period (\citealt{Cavallo:2004ta}: \citealt[142]{Cavallo:2004bf};\citealt{Klein:2002vg})
and is frequently cited in leading international publications such
as The New York times and Time Magazine as a poster child for how
neoliberal economic reforms can be implemented democratically (\citealt{Anonymous:VVaTefru}as
cited by \citealt{stokes2001public};\citealt{Silverstein:2002wm}).

Although the reform package succeeds in taming Argentina's hyperinflation
and creating economic growth, the most immediate and direct effect
of trade liberalization was an increase in unemployment, especially
among the workforce employed in previously protected, labor-intensive
industries (\citealt[10]{Beker:2011vq}). Additionally, trade liberalization
reduced the income of small-scale producers who could not compete
with cheap imports (\citealt{eckstein2001power}). Although trade
liberalization is costly to the sizable constituencies of unskilled
industrial workers and rural campesinos, the government provides little
public support for dislocated workers--reducing rather than increasing
public spending--until the government develops a targeted income-assistance
program during the currency crisis of 2001. This lack of government
responsiveness between 1990 and 2001 is puzzling given the longstanding
expectation that governments, and especially democratic regimes, must
compensate domestic losers from liberalization in order to sustain
a sufficient political coalition in favor of liberalization. This
expectation should be especially strong for democratic governments,
yet, despite the absence of government compensation, Argentina sees
relatively little domestic conflict around trade liberalization. In
fact, there are fewer strikes, strikers, and days lost to strikes
than under Alfonsin (\citealt{eckstein2001power}). Yet, dissent against
liberalization was an observable current in public discourse in the
years before Menem's repressive media regulations.%
\footnote{It is worth noting that some of the media scandals related to government
corruption were themselves linked to international economic openness.
For instance, the aggressive Argentine daily Pagina/12, whose journalists
were frequent targets of violence, sparked a scandal when they reported
on the Menem administration requiring ``substantial payment''
from the meatpacking firm Swift-Armour before they were allowed to
import machinery into the country. Another highly publicized revelation
involved the complicity of government officials in an international
drug-laundering network (Waisbord 1994).%
} Most notably in road blockages organized by protesting farmers in
1991 and 1993, unfair international competition was a recurring point
of dissent. (\citealt{McCullough:1991cs}; \citealt{Ferber:1993fb}).

If the Menem regime neglected to make provisions for its harmed constituencies,
how was it able to enact and sustain dramatic trade and capital liberalization
in a formally democratic setting? The theory presented here expects
that the Menem government is likely to repress the media in order
to silence domestic opposition to liberalization while maintaining
a formally democratic front. Consistent with the theory, the quantitative
data reveal that after a long and stable period of stable economic
openness and media freedom under Alfonsin, media freedom is volatile
immediately after Menem's liberalization begins until it is stably
repressed by 2005. According to a report \emph{Agresiones a }La Prensa\emph{
1991-1994} published by the Asociacion Madres de Plaza de Mayo, around
452 acts of aggression were committed against the press between 1991
and 1994 (\citealt{Delgado:1995tr}, as cited in \citealt[247]{Park:2002io}).
\footnote{Acts of aggression refer to ``murder, death threats, bombings, bomb
threats, intimidation, physical violence, violent threats, and termination
of broadcasts.'' If one were to count acts of excluding media from
access to the government and public name-calling of the media by the
government, the figure would be 546.%
} Although not perpetrated directly by the state, during this period
there were many acts of violence against investigative journalists
critical of the Menem regime, acts which the government denounced
but treated with impunity. (\citealt{Long:1993wb}). The Menem family's
most direct actions against media freedom were cuts to state advertising
in Pagina/12 (\citealt[27]{Waisbord:1994kq}), 11 lawsuits against
journalists under the pretense of criminal defamation (\citealt{McCullough:1991cs};
\citealt{Anonymous:TKNgfiRX}), proposals to increase libel and defamation
sentencing, and a proposal to require media outlets to purchase prohibitively
expensive libel insurance (\citealt{Sims:kgMPqAHd}).

\subsubsection{Mexico}

As in the case of Argentina, Mexico's trade liberalization in the
1990s was part of a larger national project of neoliberal economic
reform. Well before signing the North American Free Trade Agreement
(NAFTA) with the United States and Canada in 1994, Mexico unilaterally
lowered tarrifs from an average of 25\% in 1985 to 13\% by 1993 (\citealt{McDaniel:2003kw}).

Concurrent with unilateral trade liberalization and NAFTA, and as
in the case of Argentina under Menem, the Mexican government also
privatized many state-owned enterprises and eliminated many state
subsidies and price controls originally intended to support small
farmers. Most subsidies for corn and wheat producers and retail food
price controls were eliminatd by 1991 (\citealt[295]{Hufbauer:2005vh}).
In 1999, the Mexican government abolished CONASUPO, the state agency
which bought staple crops at guaranteed prices and redistributed them
to consumers (\citealt[12]{Villareal:2010vk}).

After NAFTA, US imports from the US increased from \$50.8 in 1994
to 100.4 billion in 2000 (\citealt[10]{Villareal:2010vk}). As in
the case of Argentina, trade liberalization through the 1990s hurt
small farmers and non-skilled manufacturing, as agricultural employment
decreased from 8.1 million in 1993 to 6.8 million jobs in 2003, and
value added decreased from \$32 billion to about \$25 billion in the
same period. (\citealt[289]{Hufbauer:2005vh};\citealt[14]{Villareal:2010vk}).

Also as in Argentina, increasing trade liberalization in Mexico led
to increased wage inequality between skilled and non-skilled labor.
In 1988, the real average wage level of skilled Mexican workers in
the manufacturing sector was 225\% that of non-skilled workers. In
1996, it was about 290\% that of non-skilled workers, stabilizing
until 2000 (\citealt[9]{Villareal:2010vk}) 

To support the transition into NAFTA, the government enacted the Programa
de Apoyos Directos para el Campo (Program of Direct Support for the
Countryside or ``Procampo''), which provided farmers with direct,
hectare-based income support. However, in part due to austerity following
the peso crisis, total expenditure on Procampo decreased from \$1.4
billion to \$1 billion (\citealt[295]{Hufbauer:2005vh}), despite
the price of corn in Mexico falling from \$4.84 per bushel in 1993
to \$3.65 in 1997 (\citealt[12]{Villareal:2010vk}), total number
of supported farmers decreased from 3.29 million to 2.95 million,
between 1994 and 1998 (\citealt[295]{Hufbauer:2005vh}).

On the day NAFTA went into effect in January 1994, the Ejército Zapatista
de Liberación Nacional (EZLN) launched an armed uprising in one of
Chiapas, one of Mexico's southernmost states. On that very first day
of the uprising, EZLN spokesperson Subcomandante Marcos declared NAFTA
to be ``nothing more than a death sentence to the indigenous ethnicities
of Mexico'' and their uprising to be understood as a response ``to
the decree of death that the Free Trade Agreement gives them'' (\emph{La
Journada }1994, as cited in \citealt[216]{Hayden:2009uy}).

Although Procampo helped gain support for NAFTA, immediately there
was popular discontent, such as in the Barzón Farmers movement in
Zacatecas, regarding several inadequacies of the program, including
payments not being made (\citealt[173]{Williams:2001ux}). From 1993
to 1995, Barzon movement sought and received much favorable attention
in the print media, where reporters were not under great pressure
to suppress reports (\citealt[187]{Williams:2001ux}).

Neoliberal economic reforms, including increasing trade openness,
somewhat surprinsingly in light of our expectations although not exactly
contradicting them, led to a relative opening of the domestic media
(\citealt{lawson2002building}). Between 1991 and 1993, in addition
to pursuing NAFTA as his administration's top priority, Salinas' cuts
to government spending included cutting the \emph{quid pro quo's }which
underwrote the traditional regime of media control. He specifically
ended the system of paying for reporters accomodations on presidential
trips, prohibited the distribute of bribes within the presidential
palace, reduced the government advertising in which typically functioned
as bribes for keeping media in line, ended tax deferments and credits
to media, and stopped allowing media outlets to pay their Social Security
taxes in advertisements. Privatization of state-owned enterprises
also had the effect of reducing the media's dependence on government
advertising revenues. The greater scrutiny from American and Canadian
media relaxed the domestic media environment for domestic journalists,
as it was easier for domestic journalists to report on topics that
the foreign press were already reporting on outside any control from
the Mexican government. Additionally, greater access to foreign inputs
also freed the Mexican media from an important source of government
leverage, in particular its traditional monopoly on the import of
newsprint, providing further room for the Mexican media to take risks.
As independent media outlets were gaining financial independence through
market competition, at the same time the neoliberal state was relinquishing
its traditional levers of control, led to a independence of the Mexican
media increased significantly (\citealt[76, 89]{lawson2002building}).

The international spotlight from the NAFTA negotiations also forced
Salinas to cultivate a more positive image on human rights, for instance,
when he established the National Commission for Human Rights (\citealt[107]{Dominguez:2009wd}).
Additionally, because neoliberal economic reforms actually led to
an opening of the media which the state could not control, Salinas
and after him Ernesto Zedillo moved away from traditional tactics
of media repression in favor of more modern techniques of ``news
management'' and public relations, such as controlling information
by only providing access to friendly reporters. For instance, in a
1990 press conference Salinas explicitly excluded several independent
media outlets and only permitted the most reliable pro-government
journalists. Later in 1996, the Interior Ministry for the first time
created an explicit ``blacklist'' of journalists who government
officials were supposed to not engage (\citealt[39]{lawson2002building}).

Newspaper circulation is limited in Mexico, whereas television broadcasting
dominated by Televisa is the main source of information, so it was
dominated by pro-NAFTA, pro-government ideology (\citealt{Hellman:1993wa}).
They also used it for extensive foreign media campaigning. While building
support for NAFTA, Salinas used media and PR tools extensively, including
efforts to persuade Mexican-Americans and US investors to support
NAFTA in the United States (\citealt{Morris:2001iy}). This was the
first time the Mexican government used advertising and lobbying in
its foreign relations (\citealt{Chabat:1997wj}) One of the most commented
advertisements urged US business to look to Mexico as a place where
they can hire workers for a dollar an hour. (\citealt[105]{center1993trading}
as cited in \citealt[45]{Chabat:1997wj}).

This narrative reveals dynamics which are unexpected and in a crucial
sense antithetical to our theory, for they reveal how economic liberalization
may induce greater media freedom by increasing competition and growth.
Yet, Salinas in particular was convinced that he had to protect the
government's image to succeed in his foreign economic policies (\citealt[107]{Dominguez:2009wd}).
Although economic reforms made certain kinds of repression impracticable,
under Salinas and then Zedillo the Mexican government engaged in specific
acts to exclude the press from reporting on politically sensitive
issues. Lawson observes plainly that Salinas was historically Mexico's
``undisputed master of image management'' (\citealt[39]{lawson2002building}).
Additionally, the editor of Mexico's \emph{Monitor, }José Gutiérrez-Vivó,
affirmed in 1996 that ``Salinas was the president who was hardest
on the media. He was the one who sought the most control over the
media.''(\citealt[39]{lawson2002building}) After NAFTA passes, intimidation
and direct violence against journalists at the hands of the state
can still be observed, as when the state expropriates the property
of critical editors (\citealt{OrmeJr:1997da}). In fact, despite a
de facto opening of the media due to financial independence and the
neoliberal withdrawal of the state from private enterprise, federal
state-media relations changed little until Zedillo and even his adminstration
engaged in repressive tactics such as arresting the publisher of \emph{El
Universal} for tax-related reasons in 1996. Finally, physical assault
against journalists increased throughout the period of Mexican media's
opening from 1980 to the middle of the 1990s (\citealt[81]{lawson2002building}),
which was also a period of dramatic trade opening. Most of the physical
assaults were not carried out by the government, but they were largely
treated with impunity by the government, such as in 1996, when two
journalists were murdered more ``savagely'' than ever before in
Mexico. (\citealt{Anonymous:ex})

\subsection{Summary of cases}

Thus, in the process of trade liberalization, Mexican state officials
actively seek greater control over the media as much as they can,
despite the effect of increased competition unleashing an increasingly
independent media. Both Salinas and Zedillo employed a variety of
tactics ranging from traditional repression to modern ``news management''
in order to control their image in the media during a period of rapid
trade liberalization. Salinas in particular, the earliest and most
aggressive proponent of economic liberalization in the late 1980s
and early 1990s, tried more than anyone else to control the media.


\section{Conclusion}

I find broad support for the hypothesis that trade openness is negatively
associated with media freedom because, whereas international investors
of FDI and portfolio capital prefer media freedom to media repression
in the long run, the international counterparties to foreign trade
have no such preference. Thus, although all three types of economic
openness are associated with media repression in the short run, trade
openness is associated with media repression in the short- and long-run.

The implications are important for researchers of the globalization-democracy
nexus because they highlight a specific way in which economic globalization
can generate authoritarian tendencies. The case studies reveal that
even for the most celebrated neoliberal economic reformers of the
1990s, Menem in Argentina and Salinas in Mexico, the opening of the
domestic economy was followed by government efforts to repress the
media through a variety of tactics. 

Certain unexpected findings also suggest interesting questions. For
instance, when economic liberalization increases market competition
in the media sector, it generates strong pressures which favor a free,
independent media. If this is the case, then it would provide a warrant
for expecting trade liberalization to be associated with media freedom
in the long run, as the conventional wisdom is that trade liberalization
increases market competition. Thus, if trade indeed is associated
with increased domestic competition, then it remains unclear why trade
liberalization would be statistically associated with media repression
in the long-run, a pattern evidenced in the statistical models here.
One possibility is that the domestic economic and political consequences
of trade liberalization are not fully understood and that trade liberalization
independently is not associated with increased domestic competition.


%%%%%%%%%% CHAPTER BREAK %%%%%%%%%%%%%%%%%%%%%%%%%%%%%%%%%%%%%%%%%%%%%%%%%%%%%%%

\chapter{Mass Media and the Social Construction of Globalization}

A long tradition of scholarship going back to Karl Polanyi's \emph{The Great Transformation} (\citeyear{Polanyi:2001vc}), suggests that when exposure to free trade increases, there typically follows a corresponding increase in demand for economic and social support from the state. Scholars of political science and economics have updated and extended this logic to show that exposure to the global market is often positively associated with government spending (\citealt{Adsera:2002vt, Cameron:1978vb, Garrett:1998wl,Rodrik:1998te}). However, in analysis presented here, survey data shows that demand for public intervention in response to increasing exposure to the global market is not universal; increased exposure to the global market is met, in some cases, with less demand for public intervention. This is very puzzling, given the expectations implied in the globalization-welfare argument.

	I argue that the solution to this puzzle is that individuals and groups do not perceive globalization in a simple, unmediated fashion. The media---a set of actors often neglected by IPE scholars---filter the experience of globalization. Furthermore, the owners of media outlets often have large stakes in how globalization is perceived within a country. Specifically, I hypothesize that when the state itself or foreign companies own media outlets, they will seek to represent globalization in a way that dissociates it from demands for public support. Foreign media companies will do so because their presence in the host country relies on domestic support for foreign investment and foreign ownership. State companies will do so to dampen public outcry and the demands for public support often associated with exposure to free trade.

	To demonstrate these claims, I provide evidence from three levels of analysis. I first use statistical analyses to relate conventional measures of global market exposure to attitudes regarding state intervention in the economy. My analysis uses data from the World Values Survey and covers 50 countries from 1991 to 2009. Secondly, I provide quantitative within-case evidence. A sudden transfer of media ownership in New Zealand during 2003, from mixed to strictly foreign ownership, provides an attractive opportunity to examine variation on the independent variable over time. Finally, I demonstrate that the effect of this shift in ownership can be traced at the textual level, providing qualitative evidence that the media representation of globalization is observably different before and after an increase in foreign ownership of the media company. In summary, my findings provide promising evidence---although somewhat mixed---that the degree to which globalization is met with a demand for public support is likely conditioned by the interests of the media owners responsible for constructing globalization; and that an observable change in reportage is the mechanism by which this conditioning effect is realized.

The paper proceeds in four parts. The first section provides a review of literature on the globalization-welfare nexus and a review of literature on the effects of media ownership, focusing on the untested assumptions of the former and unexplored connections between the two. The second section offers a model of the globalization-media-welfare nexus and hypotheses regarding the observable implications of the model. The third section explains the data, methodology and the triangular research design. The fourth section presents the core statistical and qualitative findings, and the fifth section concludes. Overall, the findings problematize critical assumptions in the globalization-welfare literature and provide strong evidence---although somewhat mixed---that the social construction of globalization impacts the political response to changes in global exposure.

\section{Literature Review: Globalization, Welfare, and the Media}

	Responding to the widely-held expectation that globalization implies the end of the welfare state, scholars of international relations (IR) and comparative politics have argued that exposure to global markets is positively and significantly associated with government spending (Cameron 1978; Ruggie 1983; Katzenstein 1985; Blais 1986; Rodrik 1998; Garrett 1995, 1998; Adserà and Boix 2002). The dominant theoretical explanation of this regularity is some form of a ``compensation" thesis, which suggests that in order to build a winning coalition in favor of free trade, to legitimate openness, or to hedge against external risk, governments must compensate with public support those who suffer from the opening of domestic markets. Although scholars debate particular components of this general finding, and there is an increasing effort to disentangle the specific effects of specific aspects of globalization (Burgoon 2001), there is much empirical evidence in favor of a general claim that national governments often seek compensatory domestic strategies in response to the changes and dislocations wrought by the opening of domestic markets to global exposure.
	
	Although there appears to be robust empirical evidence of the link between exposure to global markets and government spending, there is much less empirical evidence of the micro-processes supposed to explain this link. Specifically, most empirical studies of the globalization-welfare nexus implictly or explicity make two assumptions: 1) Those individuals or groups likely to be harmed by increased free trade know or believe that freer trade will cause them harm and 2) see increased government spending as a desirable compensation Rodrik 1998, 998).
	
	Very little research has sought to demonstrate either of these assumptions in particular, despite that they involve attitudinal implications one should be able to observe in survey data. Although there is a great deal of research on preferences toward free trade, there exists little research on how preferences toward domestic politics are shaped by free trade. Hays, Ehrlich, and Peinhardt (2005) observe that the micro-processes underwriting the empirics of the globalization-welfare nexus have been neglected, but their own study only tests whether social spending is, in fact, associated with support for free trade. It remains an open question whether, or under what conditions, mass attitudes in fact reflect an interpretation of globalization as necessitating a governmental response.
	
	Although empirical research on the attitudinal assumptions of the compensation thesis has been largely neglected, recent research suggests that government spending as a response to global exposure is significantly conditioned by domestic political factors more generally. For instance, Adsera and Boix show that authoritarianism is an alternative to compensation: the government may simply exclude from any consideration those who suffer from exposure to global markets (2002). Because of this, the relationship between globalization and increased spending is not as strong under authoritarian regimes. This suggests that the state is a strategic actor that will seek an alternative to the compensation strategy under certain conditions. 
	
	Students of American politics regularly study the mass media as a political institution (Hollifield 1999) having significant effects on attitudes and political behavior (Weaver 1996; Newton 1999; Druckman and Parkin 2005). Yet, for the most part, there is very little research on the relationship between the international political economy and the media. Although there exists an abiding scholarly interest in ideational notions such as social purpose and the determinants of its construction (Moravcsik 1998; Abdelal 2001; Abdelal, Blyth, Parsons 2010), only isolated qualitative research has raised the question of mass media's involvement in constructing aspects of the global political economy (Sklair 1997; Prakash 2002; Clark, Thrift, Tickell 2004). Given the growing success of institutionalist scholars seeking to understand how domestic institutions affect the aggregation of domestic interests and policy outcomes at the international level, it is suprising that the mass media have been neglected from an institutionalist perspective.
	
	Yet, extant research gives good reason to expect that media ownership in particular will have significant effects on attitudinal (and, in turn, governmental) responses to globalization. First, some Americanists find that editorial bias does indeed affect vote choice (Druckman and Parkin 2005). Second, scattered research on newspaper journalism has shown evidence that the business interests of owners are related to the direction of editorial slant. For instance, an early study by Pratt and Whiting (1986) studied editorials concerned with broadcast deregulation between 1983 and 1985 and found that newspapers owning or owned by broadcast interests were significantly more likely to editorialize in favor of broadcast deregulation. Furthermore, Ann Hollified (1999) finds that foreign ownership of newspapers is positively associated with editorials about events emanating from the country of the owner(s). If the political logic underwriting the compensation thesis is that politicians have to compensate those who lose from globalization in order to protect their electoral prospects, then the twin findings that editorial slant affects vote choice and that the interests of media owners shape media bias represent compelling \emph{prima facie} evidence that the owners of mass media play an important role in the globalization-welfare nexus.
	
	In sum, although exposure to the global market is often found to be positively associated with government spending, it remains an open empirical question whether the response of mass publics toward such exposure is, in fact, the demand or even desire for government compensation. Furthermore, although research in IPE and comparative politics suggests that domestic institutions mediate this response, the effects of the mass media have not yet been studied in this context, despite strong empirical and theoretical reasons for doing so.

\section{A Theory of Media Ownership and the Social Construction of Globalization}

	Despite the implicit assumptions of most empirical research on the globalization-welfare nexus, globalization is not a self-evident phenomena. Strictly speaking, no citizen of any country ever experiences globalization. Rather, citizens experience the effects of fairly complex economic processes rarely, if ever, observed. To the degree that exposure to international market forces takes the shape of something for which government leaders have to compensate their constituencies, such international forces have to be identified and explained to those who would suffer from them. Knowledge of and opinions regarding the effects of globalization may be determined by heuristics and cues from professional associations, trade unions, and government leaders. But arguably it is the owners and journalists of the mass media that are the most powerful set of actors charged with identifying and explaining political forces not directly observed by the public. Because the interests and incentives of media owners are not necessarily consistent with the mass publics they serve, I argue that the response of mass publics toward the global economic exposure of their home country will vary according to the different interests of different types of owners. The mechanism by which this causal connection is likely to be realized is variance in how globalization is represented in media reports.
	Different kinds of media owners are biased by different incentives and are therefore likely to represent globalization in observably different ways, ways which are marginally more likely to produce mass attitudes consistent with the owners' interests. If the standard model of the globalization-welfare literature is 

\begin{center}
$globalization \rightarrow domestic$ $policy$ $response$
\end{center}
then the model presented here argues and tests whether media ownership intervenes in this causal chain:

\begin{center}
$globalization \rightarrow media$ $ownership \rightarrow reportage \rightarrow attitudes \rightarrow$ $policy$ $response$
\end{center}

Local, non-publicly-owned media companies (LNPCs) have an interest in reporting the local costs of internationalized domestic markets. Insofar as foreign media conglomerates are better resourced, competitive rivals to LNPCs only to the degree that the domestic media market is open to foreign ownership, LNPCs have a direct interest in constructing globalization as a problem. If public sentiment toward globalization might impinge on domestic political decisions to open domestic markets, the bias of LNPCs will lean toward a construction of globalization more likely to engender protectionist sentiments than pro-free-trade attitudes. A public overly enthusiastic about opening domestic markets could be the sufficient condition for a government to permit the domination of LNPCS by foreign-owned conglomerates.

Local, publicly-owned media companies (LPCs) have strong incentives to construct globalization as relatively innocuous, or at least to blur the direct causal connection between the opening of domestic markets and the adjustment costs faced by actors in the local economy. If the compensation thesis is at all correct, then governments must be cognizant that there are direct political costs to opening domestic markets. But compensating those who lose from free trade with increased public support is only one solution, and evidently a costly one. It follows from the logic of the compensation thesis that blurring the public understanding of the causal link between free trade and its adjustment costs lessens the political necessity to provide compensation. In short, a government will not be held responsible for providing compensation if the public does not blame its woes on particular government decisions to internationalize domestic markets. Thus, media companies owned by the government have an incentive to construct globalization in a way that dissociates internationalization from its costs.

Foreign-owned media companies (FCs) have interests opposite to those of LNPCs but aligned with LPCs. Because the very right to operate and earn profits in a host country requires positive and potentially reversible political action from the host government (through often controversial legal reform on precisely this issue), foreign-owned media companies have direct stakes in public sentiment toward an open domestic market in the host country. Should accurate knowledge of the costs of globalization around the world make the public weary of foreign investment in their own country, foreign-owned media companies could very well lose access to that market altogether. This implies a powerful incentive for foreign-owned media companies to construct the phenomena of globalization as relatively innocuous. 

Predicting \emph{a priori} how bias toward globalization will manifest itself at the level of media representations is difficult. If state or foreign ownership biases media in favor of globalization, this could manifest itself as over-reportage of the benefits of globalization or an under-reportage of globalization as a contestable and controversial political development. Thus, rather than theorize about sheer volume of reportage under different ownership structures, we may theorize about the functional relationship between globalization and its reportage. That is, we would expect a perfectly unbiased media outlet to simply mirror the world, in which case there is likely to be a strong functional relationship between processes of globalization and reportage of globalization. A media outlet biased in favor of globalization may either over-report globalization positively or under-report globalization negatively, but apart from volume we would also expect to see reportage patterns functionally disconnected from real-world events. In short, bias would imply a reporting agenda (whether positively or negatively) out of sync with the fluctuation of real-world events.

\subsection{Hypotheses}
From the state of extant literature on the globalization-welfare nexus, and from the theoretical argument developed here, several hypotheses can be elucidated.

\singlespacing
\begin{quote}
H1. Other things equal, a country's increased exposure to the global market will be positively associated with preferences for compensatory government intervention. Phenomena of globalization such as foreign direct investment, importing and exporting, and financial inflows and outflows should increase the preference of mass publics for the governmental moderation of free-market consequences. This is implied in the compensation thesis.
\end{quote}
\doublespacing

\singlespacing
\begin{quote}
H2. In countries wherein foreign \emph{or} public ownership dominates media markets, the causal link between globalizing processes and preference for government intervention will be significantly less or possibly even reversed. That is, foreign or public domination of media will interact with the phenomena of globalization to reduce the latter's positive impact on preferences for government intervention. In extreme cases, foreign-public oligopoly in media markets might condition the effect of globalizing processes so much that exposure to global markets \emph{decreases} preferences toward government intervention.
\end{quote}
\doublespacing

\singlespacing
\begin{quote}
H3. Media outlets owned by foreign or public companies are less likely than LNPCs to report on globalization as a function of immediate national experience. Whereas LNPCs have an interest in covering globalization precisely to the degree that globalizing processes penetrate or threaten to further penetrate their country, FCs will cover globalization in a blanket fashion disconnected from national experience. This hypothesis is agnostic with respect to the quality or connotation of such reporting; its testable implication involves the functional relationship between quantity of phenomena and quantity of reportage.
\end{quote}
\doublespacing

\singlespacing
\begin{quote}
H4. Because they have an interest in constructing participation in the global economy as a contestable domestic policy choice, LNPCs are more likely to construct a country's experience of globalization as the outcome of a contestable domestic political decision with winners and losers, proponents and critics; state-foreign oligopolies are less likely to construct globalization as a controversial decision, and more likely to construct it as an ineluctable, objective process. In short, a qualitative analysis of media reports under different ownerships should reveal a discernibly different representation of globalization, moving from what we might term a contested change to a naturalized occurrence. This hypothesis is independent of quantity or functional relationship and involves only a qualitative difference, at the level of media representations.
\end{quote}
\doublespacing

The null hypotheses are that processes of globalization have no relationship or a negative effect on attitudes toward government intervention; that media ownership does not significantly interact with processes of globalization to affect attitudes toward government intervention; that the functional relationship between globalization and reportage is not significantly different under the specified ownership types; and that qualitative analysis is unable to identify any clear and distinct difference in the tone or content of media reports under different ownership types.

\section{Data and Method}

Because the theory developed here makes theoretical claims about a two-staged process (media representation and the response to it), and because for sampling reasons the quantitative data presented is less than ideal, I provide three empirical angles to test the theory developed above. Although these angles access the observable implications of the model at different stages and thus test for the presence of distinct relationships, the conclusions from each angle reinforce each other insofar as they represent the moving parts of one theoretical machine. For instance, if the quantitative data is consistent with the theoretical expectation that state or foreign ownership dampen the political backlash to free trade but the data preclude conclusive tests, qualitative evidence that state- or foreign-owned media are biased in favor of globalization not only supports the hypothesis of that process but also the dampening process. This is because the bias is a constituent element in the dampening process. Thus, rather than attempt the unrealistic projects of a perfectly conclusive set of statistical tests, or a soundly generalizable case study, this paper opts for an ``analytical eclecticism (Sil and Katzenstein 2010)."

\subsection{Globalization and Preferences Cross-Nationally}
	In the first half of the analysis, cross-sectional time-series regressions are presented on three measures of globalization and three measures of demand for state intervention, using data from 49 countries between 1990 and 2008. A list of countries is included as an appendix. All of the main economic data is from the World Bank Indicators and lagged by one year.  The data on preferences come from the first four waves of the World Values Survey. The data is less than ideal for all the model specifications one might like to consider, so I estimate, as the data permit, a combination of models using ordinary least-squares (OLS) with panel-corrected standard errors (PCSE) and generalized least-squares (GLS) models, with a combination of corrections for autocorrelation. All estimated models use PCSE, the conventional method for dealing with panel heteroskedasticity, or inconstant error variance across groups, and contemporaneous correlation, or spatial correlation of error terms between groups but not across time (Beck and Katz 1995). Because the data is heavily cross-sectional and there are gaps in the sample, there are only as many as 116 observations. Some missing data with a full set of controls, combined with a lagged dependent variable in some cases, leave the main models with between 63 and 43 observations. Secondary analyses, which sacrifice observations for extra statistical controls, have only 30 observations.
	
	The independent variable of exposure to globalization is measured using three conventional measures: total trade (imports plus exports over GDP), financial openness, and inward foreign direct investment. The key independent variables of interest are those representing the interaction of globalization and media ownership. I use data collected by Djankov, McLiesh, Nenova, and Shleifer (2003) on who owns the media in 97 countries around the world. For all of the countries under analysis, I identify the percentage of the media owned by the state, and the percentage owned by foreign companies, as reported by Djankov et al. Because the media ownership data pertains only to the year 1999, I make the rather tall assumption that media ownership is constant throughout the period of the sample of attitudes. Although this is obviously inaccurate, it is instrumentally justifiable as a first probe into a difficult question for which there currently exists no better longitudinal data. Furthermore, at least 1999 is nearly the median year of the sample of attitudes. Following the authors' threshold for monopoly of the media market, I construct a dummy variable Duopoly equaling one if the combined percentage of state and foreign ownership is greater than 75. The X DUOP variables are the globalization variables interacted with the Duopoly variable, modeling the effect of globalization on preferences after being run through a state-foreign media duopoly. Because the effects of ownership are likely to be greatest when the state and foreign owners together dominate the media market but we are also interested in disaggregating the effects of each type of ownership as a continuous variable, I report the results of such disaggregated models in secondary analyses.
	
	\emph{A priori}, it is not obvious what variables to control for. I include several control variables for which there might be some plausible argument, including GDP per capita, GDP growth, the unemployment rate, and inflation of the price level. The more a country is rich, growing, fully employed, and able to buy cheap goods, perhaps the more it can afford to scale back the state; or perhaps the more generous it will be with its wealth. Some argue that the institutions such as the IMF cajole countries into neoliberal philosophies; others argue the IMF engenders opposition to neoliberalism. Thus, I include the variable IMF, a measure of the total funding a country received from the IMF. The more democratic a country is, perhaps the more liberal its views. I include the variable Democracy, drawn from the Polity IV dataset by subtracting the Autocracy measure from the Democracy measure, as convention has it. Some studies show that education is associated with the preference for free trade, so it is plausible this applies on the aggregate country level and with respect to liberal domestic policies as well. The variable Education is a measure of those enrolled in tertiary education, as a percentage of the population in the relevant age range. Finally, the more state-owned enterprises a government controls, perhaps the less likely are the masses to demand more state involvement, other things equal; or, perhaps expectations are such that the masses are more likely to demand state intervention the more a government already participates in the economy. I include a variable SOE, reflecting the degree of state-ownership of enterprises, as measured by the Fraser Institute.
	
	The dependent variable, demand for state intervention, is operationalized in turn with three different measures. The World Values Survey asks several questions reflecting preferences toward the responsibilities of the state. Because some of these questions receive relatively very few answers across the world, I use three that are structured comparably and for which there are sufficiently abundant data across time and space. Each question asks the respondent to indicate on a scale from one to ten a preference between two opposite views across a liberal-interventionist continuum. The first question opposes the views that ``Private ownership of business should be increased," and ``Government ownership of business should be increased." The second opposes the views that ``The Government should take more responsibility," and ``People should take more responsibility." The third opposes the views that ``Income should be made more equal," and ``We need larger income differences as incentives." The variables corresponding to each question take the mean value for each country in each year available. Where necessary, the mean score is subtracted from one so that for each variable, higher values reflect the more liberal attitude.
	
	The data available for quantitative study of the relationship between globalization, media ownership, and attitudes around the world is inconvenient for estimating ideal models. For instance, the World Values Survey provides good indicators of the demand for state intervention, but only four waves give the data a shallow time-series dimension. Because some variables such as democracy vary little or slowly over time within particular countries, country fixed-effects and control variables of interest are often severely multi-collinear. Controlling for the serial correlation of error terms is similarly difficult because inclusion of a lagged dependent variable on the right-hand side of the equation removes as many as a third of the observations; there are often insufficient observations to compute first-order autocorrelation using the Prais-Winsten method; and in no model with a range of control variables is it possible to do both. Finally, panels are unbalanced and have gaps in the time-series. That is, some countries only have one or two observations over time (the maximum is four), and the years in which countries respond to the survey are different for each wave. This substantially complicates interpretation of time-series, cross-section regressions. I bracket the problem altogether in the first analysis and in the secondary analysis I balance the panels and remove gaps from the time series by only examining countries for which there are responses for each wave and standardizing the time code by wave rather than year. Despite all of these difficulties, inconvenient data should not deter one from learning what can be learned from it. Because limited data make it impossible to estimate perfect models, the analytical strategy adopted here is to run a series of different models using a combination of controls and diagnostic tests. Although the evidence is mixed and somewhat sensitive to model specification, some findings are more or less robust across a series of variously specified models.

Analysis begins with the methodologically loosest, most general models including a battery of control variables. These models  preclude fixed effects because of nearly time-invariant variables, and only a limited accounting of serial correlation is possible. In the models examining attitudes toward inequality and responsibility, I use a lagged dependent variable rather than the Prais-Winsten AR1 scheme because the lagged dependent variable is more likely to control for an omitted variable driving the serial correlation than the AR1 scheme, which only accounts for ``pure" serial correlation. In the models using attitudes toward privatization, I omit the lagged dependent variable (which is substantively and statistically significant) lest the loss of observations produce results unreasonable with respect to the secondary models estimated later in the paper. Because the time-series dimension is shallow compared to the cross-sectional dimension, serial correlation of errors should not be the main concern. To begin, I estimate the following model:

\vspace{2mm}
\begin{tabular}{ r c l }
  \(ATTITUDE_{it}\) & \(=\) & \(\alpha + \beta_1FDI_{it-1} + \beta_2TRADE_{it-1} +  \beta_3FIN_{it-1} + \beta_4DUOPOLY_{it-1} +\) \\
   & \(\) & \( \beta_5FDIxDUOP_{it-1} +  \beta_6TRADExDUOP_{it-1} + \beta_7FINxDUOP_{it-1} +\) \\
   & \(\) & \(\beta_8DEMOCRACY_{it-1} +  \beta_9EDU_{it-1} +   \beta_{10}INFLATION_{it-1} +\) \\
   & \(\) & \(\beta_{11}IMF_{it-1} +  \beta_{12}UNEMPLOY_{it-1} +  \beta_{13}ATTITUDE_{it-1} + e_{it}\)
\end{tabular}
\vspace{2mm}

except when attitudes toward privatization are the measure on the dependent variable, in which case I omit $\beta_{13}ATTITUDE_{it-1}$ for reasons given above. This first, most general model is vulnerable to objections and substantive questions. First, it does not account perfectly for serial correlation or unknown and omitted country-level variables. Second, it begs for a disaggregation of state versus foreign media ownership and an analysis of media effects under the duopoly threshold. In order to respond to these objections and points of interest, I estimate paired-down versions of the above model, including country fixed-effects at the cost of omitting specific control variables. Also, I disaggregate media ownership, running separate models for separate interactions. In secondary analyses, I estimate models of the form:

\vspace{2mm}
\begin{tabular}{ r c l }
  \(ATTITUDE_{it}\) & \(=\) & \(\alpha_i + \beta_1FDI_{it-1} + \beta_2TRADE_{it-1} +  \beta_3FIN_{it-1} + \beta_4PERCENTOWN_{it-1} +\) \\
   & \(\) & \( \beta_5FDIxPERCENT_{it-1} +  \beta_6TRADExPERCENT_{it-1} +\) \\
   & \(\) & \(\beta_7FINxPERCENT_{it-1} +  \beta_13ATTITUDE_{it-1} + \gamma z_i + e_{it}\) \\
\end{tabular}
\vspace{2mm}
	
	where $PERCENTOWN$ represents, depending on the model, percent of media market owned by foreign companies, state companies, and foreign plus state companies. Here panels are balanced, there are no gaps in the time code, and every model uses Prais-Winsten GLS and a lagged dependent variable to control for serial correlation. Also, $\gamma z_i$ represents a vector of country dummy variables to account for any unobserved country-level differences contributing to variation in attitudes.
	Hypothesis 1 predicts that measures of globalization will be positively assocated with demand for state intervention (i.e. negative signs for the regression coefficients, which reflect dependent variables measured in terms of neoliberal/free-market preferences). Hypothesis 2 predicts that the media interaction terms will be significant and positively signed, indicating that state-foreign domination of media markets will decrease the association between globalizing proccesses and the demand for state intervention (or, increase neoliberal/free-market attitudes).

\subsection{Within-Case Analysis of New Zealand}
	New Zealand is an ideal laboratory for examining the relationship between globalization and media ownership for at least three reasons. First, New Zealand is an exemplar of recent neoliberal globalization. The country has signed several free trade agreements (FTAs) over the past decade and the country's increasing exposure to the global economy, and in particular foreign ownership of media, has elicited heated protest from advocacy groups such as the Campaign Against Foreign Control of Aotearoa (CAFCA). For this reason, attention here is restricted to foreign ownership rather than state ownership. In short, the social conflicts brought about by globalization are most likely in a country where global exposure is very intense. This makes New Zealand an attractive test case because we are likely to find high values on measures of globalization as well as foreign ownership and thus have a good chance of identifying their interaction at work. Second, this makes New Zealand of great substantive interest for policy-makers and political actors there, where questions of foreign ownership are now of growing interest. Finally, although in one sense New Zealand is a most-likely case because values on the independent variable are rather high, in another sense it is a least-likely case for demonstrating the effects of a change in ownership over time. Because foreign ownership will be relatively high in the very first period, testing our hypotheses on a change from moderate foreign ownership to high foreign ownership is less likely to show results than studying a change from little or no foreign ownership to very high foreign ownership. If evidence is found from a relatively limited marginal shift, this should enhance confidence in the findings.
		
\subsection{A Quantitative Analysis of Independent News LTD}
	To further substantiate my argument, I provide within-case statistical analysis of newspaper reports on globalization before and after a major shift of ownership from relatively local to foreign. In July of 2003, the Australian company John Fairfax Holdings bought Independent News LTD (INL) for $1.188$ billion, acquiring all but two of New Zealand's daily newspapers with circulation above 25,000. Prior to this purchase, total foreign holdings were 49\%, with Rupert Murdoch's News Corporation having a controlling interest. With the purchase by John Fairfax Holdings, foreign ownership accounted for 100\% of the company. (Rosenberg 2002).
	Here, the independent variable is media ownership and the dependent variable is reportage of globalization. I construct a variable giving a one for each year after 2003, a zero for each year before 2003, and .5 for 2003 as a transitional year. For the dependent variable I use Lexis-Nexis searches to determine the frequency with which each of the 12 papers under the banner of INL report a story containing ``globalization" for each year available between 1995 and 2009. Again, I use cross-section time-series regression with PCSE, controlling for world trade, New Zealand trade, and inward foreign direct investment in the year under observation (financial openness is constant throughout this time period). I include dummy variables for years in which a major free-trade agreement is passed, expecting that such agreements may increase the likelihood a paper reports on globalization.
	
	Hypothesis 3 predicts that the functional relationship between processes of globalization and reportage of globalization will be greater before 2003 than after 2003, when the change toward greater foreign ownership occurs.
	
\subsection{A Qualitative Comparison of Media Representations}
	Because FTAs are typically high-profile focal points of the worldwide trend toward global free markets, a qualitative comparison of newspaper reports on comparable FTAs before and after INL's ownership change in 2003 should be a fair test of the hypothesis (H4) that foreign-owned media outlets are likely to downplay the controversial nature of globalization. New Zealand signed the New Zealand and Singapore Closer Economic Partnership (NZSCEP) in 2001, and the Trans-Pacific Strategic Economic Partnership (TPSEP) in 2005. The former includes only New Zealand and Singapore while the latter includes these two countries plus Brunei and Chile. Both are comprehensive FTAs covering trade in goods and services, rules of origin, etc. Both FTAs are between relatively small countries. Both commit New Zealand to significant tariff reductions and earn New Zealand access to overseas markets. It should be noted that New Zealand and Singapore have barely any trade restrictions at the time of signing the TPSEP, and thus the Singapore-specific effects of the TPSEP are certainly less significant than the NZSCEP. However, all things considered, it seems fair to say that if either agreement is inherently more newsworthy, it is the multilateral, multi-continental TPSEP. This controls for the possibility that we observe evidence in favor of the hypothesis only because the NZSCEP is inherently more interesting, important, or controversial. Given the difference of ownership before and after 2003, we expect that the NZSCEP will be constructed as a controversial, contestable political decision and the TPSEP will be constructed in some way, observably different, tending to neutralize the controversy and contestation typically surrounding FTAs. Precisely how this neutralization will appear is left as an open, inductive question.


\section{Findings and Discussion}
Findings are presented in three subsections. The first presents and discusses the cross-national tests of the interaction between media ownership and processes of globalization. The second examines a model of the relationship between newspaper ownership and reportage of globalization. The third provides a qualitative look at newspaper reports before and after the level of foreign ownership of the INL company changes in 2003.

\subsection{Globalization, the Media, and Preferences Cross-Nationally}

Table 1 presents results of the main model testing the interaction of media ownership and three measures of globalization. Because the average country score on an international survey question is so coarse a measure, we would expect globalization processes and media ownership to exert only relatively minor shifts. In dealing with such predicted shifts, I do not try to assess substantive effects. Any statistically significant effect, no matter how slight, is substantively significant because we are considering shifts in the mean of whole countries. Specifically, there are two key findings suggested by the first analysis. On the whole, increased exposure to the global market has a very mixed impact on attitudes reflecting the demand for compensation from the state (H1). No single component of globalization has an unambiguously significant impact on such attitudes, and in the case of capital mobility the sign changes significantly across the different questions. At the very least, these results seem to suggest that the assumption of the globalization-welfare literature that the masses respond to the opening of markets with a ``double movement" in which they hold the state responsible for providing support, is far from a necessary, automatic reality.
\pagebreak

\begin{center}
\begin{table}[htdp]
\caption{Multiple Regressions with Panel-Corrected Standard Errors}
\vspace{2em}
\begin{center}
{\small
\begin{tabular}{lccc}\hline
 & (1) & (2) & (3) \\
Independent Variables & Privatization & Inequality & Responsibility \\ \hline
\vspace{4pt} & \begin{footnotesize}\end{footnotesize} & \begin{footnotesize}\end{footnotesize} & \begin{footnotesize}\end{footnotesize} \\
fdi & -0.191** & -0.015 & -0.027 \\
\vspace{4pt} & \begin{footnotesize}(0.072)\end{footnotesize} & \begin{footnotesize}(0.069)\end{footnotesize} & \begin{footnotesize}(0.055)\end{footnotesize} \\
total trade & 0.081* & 0.020 & 0.044 \\
\vspace{4pt} & \begin{footnotesize}(0.032)\end{footnotesize} & \begin{footnotesize}(0.039)\end{footnotesize} & \begin{footnotesize}(0.037)\end{footnotesize} \\
financial openness & 4.058*** & -1.295 & -1.906** \\
\vspace{4pt} & \begin{footnotesize}(0.906)\end{footnotesize} & \begin{footnotesize}(0.741)\end{footnotesize} & \begin{footnotesize}(0.616)\end{footnotesize} \\
duopoly & -9.466 & -40.943** & -4.943 \\
\vspace{4pt} & \begin{footnotesize}(7.397)\end{footnotesize} & \begin{footnotesize}(14.354)\end{footnotesize} & \begin{footnotesize}(8.803)\end{footnotesize} \\
fdiXduop & 0.400** & 0.600 & 0.005 \\
\vspace{4pt} & \begin{footnotesize}(0.153)\end{footnotesize} & \begin{footnotesize}(0.533)\end{footnotesize} & \begin{footnotesize}(0.250)\end{footnotesize} \\
tradeXduop & -0.190*** & -0.109 & -0.064 \\
\vspace{4pt} & \begin{footnotesize}(0.051)\end{footnotesize} & \begin{footnotesize}(0.199)\end{footnotesize} & \begin{footnotesize}(0.091)\end{footnotesize} \\
financeXduop & 3.437 & 13.325* & 5.283 \\
\vspace{4pt} & \begin{footnotesize}(2.269)\end{footnotesize} & \begin{footnotesize}(6.155)\end{footnotesize} & \begin{footnotesize}(3.298)\end{footnotesize} \\
democracy & -0.051 & -3.147** & -1.385** \\
\vspace{4pt} & \begin{footnotesize}(0.507)\end{footnotesize} & \begin{footnotesize}(0.993)\end{footnotesize} & \begin{footnotesize}(0.468)\end{footnotesize} \\
edu & 0.034 & 0.124 & 0.241*** \\
\vspace{4pt} & \begin{footnotesize}(0.060)\end{footnotesize} & \begin{footnotesize}(0.090)\end{footnotesize} & \begin{footnotesize}(0.065)\end{footnotesize} \\
inflation & 0.005*** & -0.097 & -0.135** \\
\vspace{4pt} & \begin{footnotesize}(0.001)\end{footnotesize} & \begin{footnotesize}(0.051)\end{footnotesize} & \begin{footnotesize}(0.052)\end{footnotesize} \\
imf & -3.01e-10 & -1.02e-09* & -3.88e-10 \\
\vspace{4pt} & \begin{footnotesize}(4.43e-10)\end{footnotesize} & \begin{footnotesize}(4.61e-10)\end{footnotesize} & \begin{footnotesize}(3.66e-10)\end{footnotesize} \\
unemployment & -0.407** & 0.422* & -0.338 \\
\vspace{4pt} & \begin{footnotesize}(0.152)\end{footnotesize} & \begin{footnotesize}(0.186)\end{footnotesize} & \begin{footnotesize}(0.187)\end{footnotesize} \\
lagineq &  & 0.348* &  \\
\vspace{4pt} & \begin{footnotesize}\end{footnotesize} & \begin{footnotesize}(0.137)\end{footnotesize} & \begin{footnotesize}\end{footnotesize} \\
lagresp &  &  & 0.585*** \\
\vspace{4pt} & \begin{footnotesize}\end{footnotesize} & \begin{footnotesize}\end{footnotesize} & \begin{footnotesize}(0.056)\end{footnotesize} \\
Constant & 54.582*** & 51.062*** & 21.613** \\
 & \begin{footnotesize}(5.204)\end{footnotesize} & \begin{footnotesize}(14.253)\end{footnotesize} & \begin{footnotesize}(7.382)\end{footnotesize} \\
\vspace{4pt} & \begin{footnotesize}\end{footnotesize} & \begin{footnotesize}\end{footnotesize} & \begin{footnotesize}\end{footnotesize} \\
Observations & 67 & 43 & 43 \\
$R^2$ & 0.621 & 0.480 & 0.825 \\
Number of cntry & 30 & 24 & 24 \\
\multicolumn{4}{c}{\begin{footnotesize} *** p$<$0.001, ** p$<$0.01, * p$<$0.05\end{footnotesize}} \\
\multicolumn{4}{c}{\begin{footnotesize}Variables $gdpcap,$ $gdgrowth,$ and $soe$ are not
displayed.\end{footnotesize}}\\

\end{tabular}
}
\end{center}
\label{default}
\end{table}
\end{center}



\begin{figure}[htbp]
\begin{center}
\includegraphics[scale=.75]{article3_fdi.png}
\caption{{Inward FDI (\% of GDP) and Expected Preference for Privatization}}
\end{center}
\end{figure}

\begin{figure}[htbp]
\begin{center}
\includegraphics[scale=.75]{article3_finopen.png}
\caption{{Financial Openness and Preference for Inequality}}
\end{center}
\end{figure}
Perhaps most interestingly, when the media is dominated by the state or foreign companies, the effect of globalization on attitudes is significant and as predicted in two cases. For instance, other things equal, people prefer less privatization as foreign direct investment enters their country; when there is a duopoly between the state and foreign companies, actors who have common biases against the costs of globalization, then foreign-direct investment flows increase the desire for privatization. We find the same reversal with respect to the tolerance of inequality under financial openness. Although capital mobility, other things equal, does not have a statistically significant effect on the demand for reducing inequality, when mediated by a state-foreign media duopoly capital mobility is associated with favorable views regarding inequality as an incentive. Interesting as these findings are, this evidence in support of Hypothesis 2 are only two cases out of a total of nine combinations of alternative measures. In six other interaction terms, there is no statistically significant result and in one the relationship ($tradeXduop$) runs counter to Hypothesis 2. At least within the sample, in five of the nine combinations the coefficient is larger than the non-interacted globalization measures, as predicted, with state-foreign domination of the media decreasing attitudes in support of goverment intervention.

	Trade is anomalous with respect to its non-interacted and interacted terms. Surpisingly, trade seems to correlate negatively with the demand for government intervention and positively when interacted with a state-foreign media duopoly.
	
	These findings require caution. It is well known that survey respondents are extremely sensitive to question wording. On the whole, these results are far from definitive proof that the link between global market exposure and demand for public support is significantly and consistently conditioned by media ownership. Neither is it definitive proof one way or the other regarding the micro-processes of the compensation thesis. But as a first interrogation of these links, the results are certainly sufficient to conclude that the assumptions of the compensation thesis require further investigation and should not be taken for granted as in the seminal works of the globalization-welfare literature. Second, although the results regarding the media as conditioning the reception of globalization are also mixed, there is at least modest evidence for this argument. For the same reasons these results should not be interpreted as conclusively disproving the compensation thesis, the mixed evidence regarding the media's conditional effect calls for more research on this hypothesis.
	 
	 The results of secondary regressions, with country fixed-effects, controls for pure and impure serial correlation, and disaggregated media ownership, are presented below. Woolridge's test for serial correlation in panel data show that very likely serial correlation is a problem in each case except for the models using attitudes toward inequality as the dependent variable (p = 0.0066 for attitudes toward privatization; p = 0.0347 for attitudes toward responsibility). In the first of the secondary regressions, we find evidence in favor of the compensation thesis (H1) and of media's conditioning effect (H2). Independently, FDI significantly decreases attitudes favorable toward privatization in two of three cases controlling for media ownership. Trade has a similar effect in only one case, when foreign ownership is controlled for. Financial openness never has the effect predicted by the compensation thesis. The interaction of FDI with media ownership is not significant when we consider foreign ownership alone, but it is significant in the hypothesized direction when we consider state ownership and the combined effects of foreign and state ownership. Conversely, the interaction of trade and media ownership is not significant with respect to state ownership, but it is significant with respect to foreign ownership and the combined measure.
\begin{table}[htdp]
\caption{Dependent Variable: Attitudes in Favor of Privatization}
\vspace{2em}
\begin{center}
{\small
\begin{tabular}{lccc} \hline
 & (1) & (2) & (3) \\
VARIABLES & Foreign & State & Foreign + State \\ \hline
\vspace{4pt} & \begin{footnotesize}\end{footnotesize} & \begin{footnotesize}\end{footnotesize} & \begin{footnotesize}\end{footnotesize} \\
fdi & -0.05 & -0.14* & -0.17* \\
\vspace{4pt} & \begin{footnotesize}(0.050)\end{footnotesize} & \begin{footnotesize}(0.058)\end{footnotesize} & \begin{footnotesize}(0.067)\end{footnotesize} \\
ttrade & -0.12*** & -0.07 & -0.12 \\
\vspace{4pt} & \begin{footnotesize}(0.034)\end{footnotesize} & \begin{footnotesize}(0.097)\end{footnotesize} & \begin{footnotesize}(0.099)\end{footnotesize} \\
finance & 0.06 & 0.46 & 0.94 \\
\vspace{4pt} & \begin{footnotesize}(0.549)\end{footnotesize} & \begin{footnotesize}(1.351)\end{footnotesize} & \begin{footnotesize}(1.158)\end{footnotesize} \\
percentforeign & -73.94*** &  &  \\
\vspace{4pt} & \begin{footnotesize}(21.927)\end{footnotesize} & \begin{footnotesize}\end{footnotesize} & \begin{footnotesize}\end{footnotesize} \\
fdiXforeign & 0.53 &  &  \\
\vspace{4pt} & \begin{footnotesize}(0.359)\end{footnotesize} & \begin{footnotesize}\end{footnotesize} & \begin{footnotesize}\end{footnotesize} \\
tradeXforeign & 0.89* &  &  \\
\vspace{4pt} & \begin{footnotesize}(0.364)\end{footnotesize} & \begin{footnotesize}\end{footnotesize} & \begin{footnotesize}\end{footnotesize} \\
financeXforeign & 1.97 &  &  \\
\vspace{4pt} & \begin{footnotesize}(3.217)\end{footnotesize} & \begin{footnotesize}\end{footnotesize} & \begin{footnotesize}\end{footnotesize} \\
percentstate &  & -31.32 &  \\
\vspace{4pt} & \begin{footnotesize}\end{footnotesize} & \begin{footnotesize}(20.004)\end{footnotesize} & \begin{footnotesize}\end{footnotesize} \\
fdiXstate &  & 0.79* &  \\
\vspace{4pt} & \begin{footnotesize}\end{footnotesize} & \begin{footnotesize}(0.330)\end{footnotesize} & \begin{footnotesize}\end{footnotesize} \\
tradeXstate &  & 0.26 &  \\
\vspace{4pt} & \begin{footnotesize}\end{footnotesize} & \begin{footnotesize}(0.250)\end{footnotesize} & \begin{footnotesize}\end{footnotesize} \\
financeXstate &  & 2.85 &  \\
\vspace{4pt} & \begin{footnotesize}\end{footnotesize} & \begin{footnotesize}(4.358)\end{footnotesize} & \begin{footnotesize}\end{footnotesize} \\
percentown &  &  & -39.64** \\
\vspace{4pt} & \begin{footnotesize}\end{footnotesize} & \begin{footnotesize}\end{footnotesize} & \begin{footnotesize}(14.150)\end{footnotesize} \\
fdiXfs &  &  & 0.66** \\
\vspace{4pt} & \begin{footnotesize}\end{footnotesize} & \begin{footnotesize}\end{footnotesize} & \begin{footnotesize}(0.232)\end{footnotesize} \\
tradeXfs &  &  & 0.40* \\
\vspace{4pt} & \begin{footnotesize}\end{footnotesize} & \begin{footnotesize}\end{footnotesize} & \begin{footnotesize}(0.201)\end{footnotesize} \\
financeXfs &  &  & -0.56 \\
\vspace{4pt} & \begin{footnotesize}\end{footnotesize} & \begin{footnotesize}\end{footnotesize} & \begin{footnotesize}(1.602)\end{footnotesize} \\
lagpriv & 0.80*** & 0.75*** & 0.70*** \\
\vspace{4pt} & \begin{footnotesize}(0.063)\end{footnotesize} & \begin{footnotesize}(0.134)\end{footnotesize} & \begin{footnotesize}(0.097)\end{footnotesize} \\
Constant & 15.47*** & 15.18 & 21.27** \\
 & \begin{footnotesize}(3.022)\end{footnotesize} & \begin{footnotesize}(9.043)\end{footnotesize} & \begin{footnotesize}(7.894)\end{footnotesize} \\
\vspace{4pt} & \begin{footnotesize}\end{footnotesize} & \begin{footnotesize}\end{footnotesize} & \begin{footnotesize}\end{footnotesize} \\
Observations & 30 & 30 & 30 \\
$R^2$ & 0.86 & 0.85 & 0.87 \\
Number of cntry & 10 & 10 & 10 \\
 $Rho$ & -.07 & -.19 & -.18 \\ \hline
\multicolumn{4}{c}{\begin{footnotesize} Standard errors in parentheses.\end{footnotesize}} \\
\multicolumn{4}{c}{\begin{footnotesize} *** p$<$0.001, ** p$<$0.01, * p$<$0.05\end{footnotesize}} \\
\multicolumn{4}{c}{\begin{footnotesize} Coefficients for country fixed-effects are not displayed. \end{footnotesize}} \\

\end{tabular}
}
\end{center}
\label{default}
\end{table}


\begin{table}[htdp]
\caption{Dependent Variable: Attitudes in Favor of Inequality}
\vspace{2em}
\begin{center}
{\small
\begin{tabular}{lccc} \hline
 & (1) & (2) & (3) \\
VARIABLES & Foreign & State & Foreign+State \\ \hline
\vspace{4pt} & \begin{footnotesize}\end{footnotesize} & \begin{footnotesize}\end{footnotesize} & \begin{footnotesize}\end{footnotesize} \\
fdi & -0.18 & -0.01 & -0.13 \\
\vspace{4pt} & \begin{footnotesize}(0.122)\end{footnotesize} & \begin{footnotesize}(0.108)\end{footnotesize} & \begin{footnotesize}(0.125)\end{footnotesize} \\
ttrade & 0.11 & 0.07 & 0.09 \\
\vspace{4pt} & \begin{footnotesize}(0.121)\end{footnotesize} & \begin{footnotesize}(0.073)\end{footnotesize} & \begin{footnotesize}(0.109)\end{footnotesize} \\
finance & 1.96** & 1.61* & 2.68*** \\
\vspace{4pt} & \begin{footnotesize}(0.634)\end{footnotesize} & \begin{footnotesize}(0.630)\end{footnotesize} & \begin{footnotesize}(0.659)\end{footnotesize} \\
percentforeign & 23.39 &  &  \\
\vspace{4pt} & \begin{footnotesize}(39.784)\end{footnotesize} & \begin{footnotesize}\end{footnotesize} & \begin{footnotesize}\end{footnotesize} \\
fdiXforeign & 0.38 &  &  \\
\vspace{4pt} & \begin{footnotesize}(0.672)\end{footnotesize} & \begin{footnotesize}\end{footnotesize} & \begin{footnotesize}\end{footnotesize} \\
tradeXforeign & -0.14 &  &  \\
\vspace{4pt} & \begin{footnotesize}(0.687)\end{footnotesize} & \begin{footnotesize}\end{footnotesize} & \begin{footnotesize}\end{footnotesize} \\
financeXforeign & -20.56*** &  &  \\
\vspace{4pt} & \begin{footnotesize}(5.998)\end{footnotesize} & \begin{footnotesize}\end{footnotesize} & \begin{footnotesize}\end{footnotesize} \\
percentstate &  & 52.32 &  \\
\vspace{4pt} & \begin{footnotesize}\end{footnotesize} & \begin{footnotesize}(29.819)\end{footnotesize} & \begin{footnotesize}\end{footnotesize} \\
fdiXstate &  & -1.28 &  \\
\vspace{4pt} & \begin{footnotesize}\end{footnotesize} & \begin{footnotesize}(0.852)\end{footnotesize} & \begin{footnotesize}\end{footnotesize} \\
tradeXstate &  & -0.47 &  \\
\vspace{4pt} & \begin{footnotesize}\end{footnotesize} & \begin{footnotesize}(0.351)\end{footnotesize} & \begin{footnotesize}\end{footnotesize} \\
financeXstate &  & -13.79** &  \\
\vspace{4pt} & \begin{footnotesize}\end{footnotesize} & \begin{footnotesize}(5.093)\end{footnotesize} & \begin{footnotesize}\end{footnotesize} \\
percentown &  &  & 10.59 \\
\vspace{4pt} & \begin{footnotesize}\end{footnotesize} & \begin{footnotesize}\end{footnotesize} & \begin{footnotesize}(16.050)\end{footnotesize} \\
fdiXfs &  &  & -0.08 \\
\vspace{4pt} & \begin{footnotesize}\end{footnotesize} & \begin{footnotesize}\end{footnotesize} & \begin{footnotesize}(0.406)\end{footnotesize} \\
tradeXfs &  &  & -0.12 \\
\vspace{4pt} & \begin{footnotesize}\end{footnotesize} & \begin{footnotesize}\end{footnotesize} & \begin{footnotesize}(0.294)\end{footnotesize} \\
financeXfs &  &  & -8.14*** \\
\vspace{4pt} & \begin{footnotesize}\end{footnotesize} & \begin{footnotesize}\end{footnotesize} & \begin{footnotesize}(1.276)\end{footnotesize} \\
lagineq & 0.19 & 0.03 & 0.08 \\
\vspace{4pt} & \begin{footnotesize}(0.208)\end{footnotesize} & \begin{footnotesize}(0.180)\end{footnotesize} & \begin{footnotesize}(0.184)\end{footnotesize} \\
Constant & 40.80*** & 49.30*** & 46.47*** \\
 & \begin{footnotesize}(11.870)\end{footnotesize} & \begin{footnotesize}(11.757)\end{footnotesize} & \begin{footnotesize}(11.794)\end{footnotesize} \\
\vspace{4pt} & \begin{footnotesize}\end{footnotesize} & \begin{footnotesize}\end{footnotesize} & \begin{footnotesize}\end{footnotesize} \\
Observations & 30 & 30 & 30 \\
$R^2$ & 0.58 & 0.37 & 0.37 \\
Number of cntry & 10 & 10 & 10 \\
 $Rho$ & -.15 & .12 & -.03 \\ \hline
\multicolumn{4}{c}{\begin{footnotesize} Standard errors in parentheses\end{footnotesize}} \\
\multicolumn{4}{c}{\begin{footnotesize} *** p$<$0.001, ** p$<$0.01, * p$<$0.05\end{footnotesize}} \\
\multicolumn{4}{c}{\begin{footnotesize} Coefficients for country fixed-effects are not displayed. \end{footnotesize}} \\
\end{tabular}
}
\end{center}
\label{default}
\end{table}



Results for attitudes toward inequality show no evidence in favor of either the compensation thesis or the media interaction hypothesis. It is possible that attitudes towards whether the government should intervene to lessen inequality reflect more than the demand for state intervention in general, more so than attitudes toward privatization.

When using measures of attitudes toward personal versus government responsibility as the dependent variable, we find evidence that FDI significantly decreases the attitude that people should be more responsible, but evidence that trade either has no significant effect or it increases the belief in personal responsibility in the case that we control for foreign media ownership. More interestingly, we find evidence that state ownership and foreign plus state ownership significantly condition the effect that FDI and trade have on attitudes toward who should be more responsible. Again, the effects are in the predicted direction, with state ownership reversing the effect of two globalization processes, from increasing the belief that the government should be more responsible to increasing beliefs that individuals should be more responsible instead.


\begin{table}[htdp]
\caption{Dependent Variable: Attitudes in Favor of Personal Responsibility}
\vspace{2em}
\begin{center}
{\small
\begin{tabular}{lccc} \hline
 & (1) & (2) & (3) \\
VARIABLES & Foreign & State & Foreign+State \\ \hline
\vspace{4pt} & \begin{footnotesize}\end{footnotesize} & \begin{footnotesize}\end{footnotesize} & \begin{footnotesize}\end{footnotesize} \\
fdi & -0.19* & -0.34*** & -0.22** \\
\vspace{4pt} & \begin{footnotesize}(0.087)\end{footnotesize} & \begin{footnotesize}(0.060)\end{footnotesize} & \begin{footnotesize}(0.082)\end{footnotesize} \\
ttrade & 0.21* & 0.12 & 0.08 \\
\vspace{4pt} & \begin{footnotesize}(0.095)\end{footnotesize} & \begin{footnotesize}(0.096)\end{footnotesize} & \begin{footnotesize}(0.122)\end{footnotesize} \\
finance & 0.91 & 2.82*** & 2.03* \\
\vspace{4pt} & \begin{footnotesize}(0.551)\end{footnotesize} & \begin{footnotesize}(0.573)\end{footnotesize} & \begin{footnotesize}(0.796)\end{footnotesize} \\
percentforeign & -37.50 &  &  \\
\vspace{4pt} & \begin{footnotesize}(36.474)\end{footnotesize} & \begin{footnotesize}\end{footnotesize} & \begin{footnotesize}\end{footnotesize} \\
fdiXforeign & 1.00 &  &  \\
\vspace{4pt} & \begin{footnotesize}(0.573)\end{footnotesize} & \begin{footnotesize}\end{footnotesize} & \begin{footnotesize}\end{footnotesize} \\
tradeXforeign & 0.17 &  &  \\
\vspace{4pt} & \begin{footnotesize}(0.654)\end{footnotesize} & \begin{footnotesize}\end{footnotesize} & \begin{footnotesize}\end{footnotesize} \\
financeXforeign & -4.16 &  &  \\
\vspace{4pt} & \begin{footnotesize}(4.048)\end{footnotesize} & \begin{footnotesize}\end{footnotesize} & \begin{footnotesize}\end{footnotesize} \\
percentstate &  & -101.88*** &  \\
\vspace{4pt} & \begin{footnotesize}\end{footnotesize} & \begin{footnotesize}(28.788)\end{footnotesize} & \begin{footnotesize}\end{footnotesize} \\
fdiXstate &  & 1.96** &  \\
\vspace{4pt} & \begin{footnotesize}\end{footnotesize} & \begin{footnotesize}(0.621)\end{footnotesize} & \begin{footnotesize}\end{footnotesize} \\
tradeXstate &  & 0.80** &  \\
\vspace{4pt} & \begin{footnotesize}\end{footnotesize} & \begin{footnotesize}(0.298)\end{footnotesize} & \begin{footnotesize}\end{footnotesize} \\
financeXstate &  & -9.90** &  \\
\vspace{4pt} & \begin{footnotesize}\end{footnotesize} & \begin{footnotesize}(3.775)\end{footnotesize} & \begin{footnotesize}\end{footnotesize} \\
percentown &  &  & -52.40** \\
\vspace{4pt} & \begin{footnotesize}\end{footnotesize} & \begin{footnotesize}\end{footnotesize} & \begin{footnotesize}(17.727)\end{footnotesize} \\
fdiXfs &  &  & 0.82** \\
\vspace{4pt} & \begin{footnotesize}\end{footnotesize} & \begin{footnotesize}\end{footnotesize} & \begin{footnotesize}(0.296)\end{footnotesize} \\
tradeXfs &  &  & 0.57* \\
\vspace{4pt} & \begin{footnotesize}\end{footnotesize} & \begin{footnotesize}\end{footnotesize} & \begin{footnotesize}(0.280)\end{footnotesize} \\
financeXfs &  &  & -4.58* \\
\vspace{4pt} & \begin{footnotesize}\end{footnotesize} & \begin{footnotesize}\end{footnotesize} & \begin{footnotesize}(2.096)\end{footnotesize} \\
lagresp & 0.71*** & 0.29 & 0.55*** \\
\vspace{4pt} & \begin{footnotesize}(0.142)\end{footnotesize} & \begin{footnotesize}(0.153)\end{footnotesize} & \begin{footnotesize}(0.105)\end{footnotesize} \\
Constant & 7.89 & 32.69*** & 21.10*** \\
 & \begin{footnotesize}(4.789)\end{footnotesize} & \begin{footnotesize}(9.112)\end{footnotesize} & \begin{footnotesize}(5.687)\end{footnotesize} \\
\vspace{4pt} & \begin{footnotesize}\end{footnotesize} & \begin{footnotesize}\end{footnotesize} & \begin{footnotesize}\end{footnotesize} \\
Observations & 30 & 30 & 30 \\
$R^2$ & 0.60 & 0.62 & 0.75 \\
Number of cntry & 10 & 10 & 10 \\
 $Rho$ & -.08 & -.01 & -.19 \\ \hline
\multicolumn{4}{c}{\begin{footnotesize} Standard errors in parentheses\end{footnotesize}} \\
\multicolumn{4}{c}{\begin{footnotesize} *** p$<$0.001, ** p$<$0.01, * p$<$0.05\end{footnotesize}} \\
\multicolumn{4}{c}{\begin{footnotesize} Coefficients for country fixed-effects are not displayed. \end{footnotesize}} \\
\end{tabular}
}
\end{center}
\label{default}
\end{table}

\pagebreak

\subsection{A Quantitative Analysis of Independent News LTD}
Reported below are the regression results analyzing the relationship between globalizing processes and mentions of globalization in 12 INL newspapers before and after their purchase by foreign-owned John Fairfax Holdings. The variable of interest is the first variable reported, the interaction term representing New Zealand trade filtered through Fairfax ownership. The significant and negatively signed coefficient suggests that under Fairfax, the papers' reportage of globalization was significantly less an increasing function of New Zealand trade than before Fairfax.

\pagebreak
\begin{center}
\begin{table}[htdp]
\caption{Multiple Regression with Panel-Corrected Standard Errors}
\vspace{2em}

\begin{center}
{\small
\begin{tabular}{lc} \hline
 &  \\
Independent Variables & Coefficient \\ \hline
\vspace{4pt} & \begin{footnotesize}\end{footnotesize} \\
tradeXfairfax & -110.45* \\
\vspace{4pt} & \begin{footnotesize}(43.945)\end{footnotesize} \\
nztrade & 135.36*** \\
\vspace{4pt} & \begin{footnotesize}(21.069)\end{footnotesize} \\
fairfax & 67.55** \\
\vspace{4pt} & \begin{footnotesize}(25.692)\end{footnotesize} \\
wtrade & -0.00 \\
\vspace{4pt} & \begin{footnotesize}(0.000)\end{footnotesize} \\
fdi & -13.98 \\
\vspace{4pt} & \begin{footnotesize}(16.475)\end{footnotesize} \\
thailand & -4.69** \\
\vspace{4pt} & \begin{footnotesize}(1.708)\end{footnotesize} \\
asean & 0.09 \\
\vspace{4pt} & \begin{footnotesize}(3.405)\end{footnotesize} \\
china & -2.30 \\
\vspace{4pt} & \begin{footnotesize}(2.661)\end{footnotesize} \\
singapore & -8.58*** \\
\vspace{4pt} & \begin{footnotesize}(1.970)\end{footnotesize} \\
lag & 0.25* \\
\vspace{4pt} & \begin{footnotesize}(2.258)\end{footnotesize} \\
Constant & -67.82*** \\
 & \begin{footnotesize}(12.599)\end{footnotesize} \\
\vspace{4pt} & \begin{footnotesize}\end{footnotesize} \\
Observations & 153 \\
Number of newspapers & 12 \\
$R^2$ & 0.86 \\
 Adj. $R^2$ & . \\ \hline
\multicolumn{2}{c}{\begin{footnotesize} Includes fixed effects by newspaper. \end{footnotesize}} \\
\multicolumn{2}{c}{\begin{footnotesize} *** p$<$0.001, ** p$<$0.01, * p$<$0.05\end{footnotesize}} \\
\end{tabular}
}
\end{center}

\label{default}
\end{table}
\end{center}

This is evidence in favor of the foreign-ownership part of Hypothesis 3, namely that the bias of foreign-ownership is likely to imply a functional disconnect between real-world, national economic integration in the global economy and reportage of globalization. Graphically, it is easy to observe that when one runs separate regressions for the years 1995-2003 and 2003-2009, the slope is significantly steeper and there is less standard error in the first period. The graph is only interesting descriptively because the confidence intervals are inflated by separating the data into two samples and running two separate regressions. The significance of the interaction term in the full model shows that the difference is significant. I argue that this is modest evidence in favor of the hypothesis that foreign-owned media outlets are likely to be biased toward globalization in a way that represents it in a way disconnected from the real, lived experiences of a particular national public.

\begin{center}
\begin{figure}[htbp]
\includegraphics[scale=.85]{article3_rdgraph.png}
\caption{{Discontinuous Regressions for Reportage of  Globalization and New Zealand Trade, 1995-2009
}}
\end{figure}
\end{center}

\subsection{A Qualitative Comparison of Media Representations}

To begin analysis of the actual social construction of globalization, consider the following table which summarizes total number of news stories reporting on two major FTAs signed by New Zealand in the past decade. Content analysis of newspaper reports in these time spans considered an article to be "negative/debatable" if it mentioned any viewpoint critical of globalization or if it mentioned that globalization is debated. Immediately, one notices that in the year leading up to an agreement, the year of an agreement, and the year after an agreement, news reports under Fairfax are relatively few, late, and uncritical.

\begin{center}
\begin{table}[htdp]
\caption{Comparison of TPSEP and NZSCEP Media Reports}
\vspace{2em}
\begin{center}
{\small
\begin{tabular}{lcccc} \hline
& $T_{-1}$ &  $T$ &  $T_{+1}$ & Total \\ \hline
& \begin{footnotesize}\end{footnotesize} & \begin{footnotesize}\end{footnotesize} & \begin{footnotesize}\end{footnotesize} \\
\bf{TPSEP} & & & &\\
Yearly Total & 0 & 0 & 3 & 3\\
\# Negative/Debatable & 0 & 0 & 0 & 0 \vspace{4pt} \\
\bf{NZSCEP} & & & & \\
Yearly Total & 0 & 7 & 1 & 8\\
\# Negative/Debatable & 0 & 6 & 1 & 7 \vspace{10pt}\\
\end{tabular}
}
\end{center}
\label{default}
\end{table}
\end{center}


More specifically, content analysis reveals the following distinct pattern of representation. After 2003, one finds in the pages of INL newspapers a \emph{naturalization} of globalization. In the reports on the 2001 NZSCEP, globalization is a political phenomenon with proponents and critics debating the issue. For instance, six of the reports are ``Day in the House" articles reporting that the legislature was actively reviewing and debating the proposed NZSCEP. The NZSCEP is an uncertain prospect with costs and benefits, as well as proponents (Labour and National parties) and opponents (Alliance and Green parties) (Evening Post, Nov. 16, 2000). In contrast, after 2006, in the three stories reporting on the TPSEP, it is striking that in each case, the defining problematic which makes the story newsworthy is that New Zealand is suffering in one way or another from \emph{not enough} free trade. (Dominion Post 2006; Timaru Herald 2006; The Press 2006). A representative example is the second sentence in the 2006 article from \emph{The Press}, which is supposed to capture the interest of the reader with "growing fears New Zealand may be losing out in the race to clinch bilateral deals to remove trade barriers such as tariffs and quotas." Noticeably, Lexis-Nexis finds zero "Day in the House" articles such as those that represented the NZSCEP as a legislative debate. In 2006, globalization becomes a given, something that \emph{happened}. Thus, a strikingly noticeable qualitative difference in the social construction of globalization emerges after a shift toward foreign ownership: globalizing free-trade agreements that were previously represented as contentious, debatable, prospective political issues become already decided, hardly arguable, given or naturalized facts.


\section{Conclusion}

To conclude, I find mixed but suggestive evidence, from three levels of analysis, that media ownership significantly conditions the political response of mass publics toward globalization. The main quantitative analysis reveal that the assumptions of the compensation thesis are problematic: in relatively few of the different model specifications examining different measures of globalization and different attitudes toward government intervention was there significant evidence that people demand government intervention to compensate for exposure to global free trade. In relatively few cases was the sign of the coefficient even as predicted by this thesis. The main findings of interest, and the main potential contribution of this paper, relate to the effect of media ownership in mediating the political response to exposure to global free trade. Although findings were not consistent and were very sensitive to model specification, more than half of the total cross-national models showed that either foreign or state ownership significantly dampened or reversed the effect of some globalization process on some attitude toward the demand for state intervention. Given the shortcomings of the data, I interpret the sum of the findings as a promising warrant for more research on this question. In short, the findings problematize accepted wisdom and provide a suggestive first cut at a fairly new, fairly bold hypothesis. The within-case quantitative and qualitative evidence further suggest that media ownership significantly determines bias in the social construction of globalization.


\section{Appendix}

\begin{table}[htdp]
\caption{Appendix: Countries in Main Quantitative Analysis}
\begin{center}
\vspace{2em}
\begin{tabular}{ccc}
Albania & Algeria & Argentina \\ \\
Armenia & Australia & Azerbaijian \\ \\
Bangladesh & Belarus & Belgium \\ \\
Bosnia Herzegovina & Brazil & Bulgaria \\ \\
Canada & Chile & China \\ \\
Colombia & Croatia & Czech Republic \\ \\
Dominican Republic & Egypt & El Salvador \\ \\
Estonia & Finland & France \\ \\
Georgia & Ghana & Guatemala \\ \\
Hungary & Iceland & India \\ \\
Ireland & Italy & Japan \\ \\
Korea & Mexico & Netherlands \\ \\
New Zealand & Norway & Poland \\ \\
Slovenia & Spain & Sweden \\ \\
Switzerland & Turkey & United Kingdom \\ \\
United States & Venezuela \\ \\


\end{tabular}
\end{center}
\label{default}
\end{table}%


\bibliographystyle{apsr2006}

\bibliography{mybib}

\end{document}