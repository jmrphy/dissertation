\section{The Lost Tradition of Media as Propaganda}

From the 1920s until the end of World War II, the conventional wisdom was that the role of mass
media in modern society was, and ought to be, an instrument of propaganda for the optimal
functioning of the state (\citealt{Bernays:2004vo, lippmann1932public}). With the rise of
the mathematical theory of communication and what would more broadly become known as information theory, (\citealt{Shannon:2013iy};\citealt{gleick2011information}),the post-war period saw a flowering of social-scientific
efforts to link the propaganda role of media to this new framework (\citealt{wiener1965cybernetics, Deutsch:1953ww, Deutsch:1966ux, McLuhan:1994tf,
Ellul:1965uf}). Additionally, there were also parallel movements in radical, continental
theory \citet{Horkheimer:2009te}, \citet{adorno2001culture}, \citet{Debord:1967vn}.} Eventually, however, these currents were superceded by the Colombia and Michigan schools, both of which tended to stress the limitations of media effects, and would ultimately constitute the baseline for most future social science research into media effects.

Today, these incipient social-scientific theories of the media appear remarkably cynical: given the
longstanding conventional wisdom of elites that the media were mere tools of propaganda, the
emergence of legitimately scientific models of information quite naturally led social theorists to
conceptualize the media as instruments of social control. Thus, these early efforts are laden with
surprisingly sinister vocabularies, the most recurring themes revolving around the control,
monitoring, and shaping of mass publics.

These early social scientists of the media, most of whom were writing within democratic states, had
surprisingly little to say about the media having anything to do with the empowerment of the masses.
This is puzzling given that, today, scholars and schoolchildren alike are most immediately inclined
to think of the media as a ``watchdog" over government, the main purpose of which is to ensure
popular sovereignty through the free flow of information. Today, even those social scientists most
critical of media effects are exceptions which prove the rule, as they typically frame their
findings as raising questions about the media's well-known function as government watchdog.

If the earliest and most influential social-scientific models of the media were so cynical, then
why, when and, how did contemporary social scientists develop such a sanguine view of the media? I
do not pretend to offer any definitive answer to such questions, as the present volume is a social-
scientific study of media effects in international political economy, not an intellectual history or
sociology of ideas. Yet, it is necessary to float some short and provisional answers to these
questions because the studies which follow are motivated and informed by certain intuitions and
provisional hypotheses regarding this peculiar puzzle in the history of the social sciences. And
substantively, the studies here do begin to fill the gap between the early cynicism of media
scholars and the sanguinity of contemporary social scientists. However, as is necessary in social
science, to make the general, overarching hypotheses analytically tractable, each of the studies in
this volume have so narrowly operationalized them that they literally would be invisible were I to
not state them up front. Thus, to provide helpful background but also to adequately situate the
importance of this volume, it is worthwhile to sketch some of the more general and ambitious
hypotheses for which this volume does offer some provisional evidence, however much it is only a
beginning.

So where did our benign democratic notions of the media come from, when for several decades the
conventional wisdom on the role of media was expressly anti-democratic and the very nature of
information was now becoming understood scientifically? This intersection in intellectual history
would seem to predict a future in which the various media would become all the more powerful tools
for small national elites to control and manipulate mass publics, the vision largely shared by so
many incipient post-war social scientists. But yet, the notion of propaganda recedes from the social
sciences from its high point around 1950, while the study of information continues increasing and
the social scientific study of media begins in earnest.

In some sense, these social-scientific currents which are only just beginning to theorize the mass
media with an emphasis on propaganda and information control are absorbed by government and the
private sector. It appears as if this incipient social-scientific perspective is adopted and
\emph{put into practice} by various branches of the state, as in the rise of ``counterinsurgency''
abroad and government ``public relations'' at home, or otherwise the private commercial development
of communications engineering and ``operations management.'' As the new sciences of information
control are put into practice by the state and the private sector, it is at this time that the
curiously mild-mannered attitude toward the media is elevated into a baseline for political science
research (Lazarsfeld 1944; Berelson 1954; Campbell et al. 1960).\footnote{The Columbia group
around Lazarsfeld, from the beginning, was really only interested in what was already a highly
narrow and market-oriented type of behavioral variation. Bartels notes how they only turned to
electoral behavior when they could not find grant money to study consumer behavior (Rossi 1959,
15-16, as cited in Bartels 2008). The point for our purposes is that these pioneering studies which
would become baselines for the modern study of political behavior rose to prominence with a view of
the media that already abstracted away from the more ``sinister'' media effects predicted by the
group discussed above. Thus, by the 1954 \emph{Voting: A Study of Opinion Formation in a
Presidential Campaign, }the Columbia group finds little evidence for the role of parties and media
in presidential campaigns. The later Michigan group, whose election studies would become an
increasingly institutionalized backbone of American political science research, also had a view of
the media which is puzzlingly inane and conservative read alongside work of roughly the same period
by someone such as Karl Deutsch. Of course, I do not here take issue with the validity of these
early findings as far as they go; my point is only to flag that these foundational studies in
American political science adopted an approach which generated strikingly inane findings on the
political effects of media, especially when read alongside those who were grappling with the more
general historical functions of media as institutions of social control.} 

### THIS IS WHAT ORFEO SAYS CAN STAY IN
This baseline
conventional wisdom of ``limited effects'' from media would no doubt be challenged within political
science, but it nonetheless has remained dominant (Katz; Bartels 2008). That research funding
distributed by the U.S. government and the private sector played a prominent role in the
mainstreaming and institutionalization of the Columbia and Michigan models of political science
research approaches, at the very same time that information theory is being rapidly mobilized in
actual state and corporate operations and logistics, further tempts one to the hypothesis that
perhaps the greatest achievement of state and corporate media control was to have ensured that
social scientists would never quite succeed in understanding or demonstrating the media's function
in social control.

This is why the present detour through intellectual history is not merely an overly general
literature review; rather, this somewhat sociological review of the literature is itself suggestive
empirical evidence, however circumstantial, regarding a crucial transformation in our thinking and
practices of media politics. I have traced in the record of the social sciences the transformation
of media-as-propaganda to media-as-transparency to outline a general gap in the literature which
this volume contributes to filling, but also to present some provisional evidence, very close to
home, of precisely how media-as-propaganda may shape certain institutional political outcomes in
ways which have hardly been noticed. Indeed, if the media are most importantly propaganda tools
then, to the very degree they are politically effective, we would expect them to go unnoticed by
institutional social science. Indeed, even the exceptions suggest evidence for this rule, for the
most popular intellectual inheritors of the media-as-propaganda tradition today remain marginal to
dominant mainstream social science.\footnote{For instance, see
\citealt{Herman:1988ta,mcchesney2000rich,luhmann2000reality}}


\section{Globalization and its Variable Discontents}

These peculiar transformations in the study and practice of media politics curiously precede the
period of remarkable, worldwide economic integration which has come to be known as ``globalisation."
Globalisation typically refers to the period from the early 1970s to the present, when countries
around the world saw large increases in the flows of goods, services, and capital across borders. It
is widely thought by political economists that economic integration is welfare-improving on net and
in the long-run, yet even staunch ``free market" economists acknowledge that economic integration
raises the incomes of certain domestic groups and lowers the incomes of other groups, at least in
the short-run. For this reason, globalisation has brought with it many notable examples of political
protest and social unrest. Yet, discontent around economic globalisation has varied across time and
space and there remains much debate regarding the conditions under which domestic political
processes respond to economic globalisation in different ways.

Both the concept and the processes of globalization have had a dubious impact on the popular
political imagination. The ideology which this very concept bears witness to is that globalisation
is a process, a noun, something which has descended on the system of nation-states from the outside,
causing rather than being caused by the actions of policymakers. As such, the very concept
represents a political bias because the casting of human actions as a process, the replacement of
verbs with nouns ending in ``-ation'', is a tendency higlighted by critical discourse analysts of
authority figures seeking to obscure the reality of their power (\citealt{fowler1979language}, 33-41).
\footnote{Interestingly, it is also a tendency of social scientists (\citealt{billig2013learn}, 117).} It is well
documented that politicians strategically deploy the rhetoric of globalization to justify economic
reform (\citealt{Hay:2011dh}), and it has also been shown that the rise of economic globalization
weakens the tendency of voters to hold politicians accountable for economic performance
(\citealt{Hellwig:2007gn,Hellwig:2007jr}).

Thus, the economics, rhetoric, and politics of economic globalisation since the 1970s appear at
first glance quite consistent with the economics, rhetoric, and politics of the media at that time.
If the disappearance of propaganda theories during and after the war represented, as I argued above,
not the decline of that idea but rather its implementation, then it would stand to reason that the
role of media in promoting state interests appears to have played a role in the popular and
scholarly narrative of economic globalisation since the 1970s. Specifically, the overarching thesis
of this volume--which remains too abstract and provisional to permit rigorous direct testing, but
which the present studies begin to make tractable--is that the rise of modern media around the world
has, in different ways, helped state elites to promote certain perceptions of international
integration to fundamentally undemocratic ends.
